\documentclass[10pt]{beamer}
\usetheme{jambro}

\title[]{Métodos Quantitativos em Economia I - Apresentação}
\author[]{Paulo Victor da Fonseca}
\date{01 de março de 2023}

\hypersetup{
    colorlinks = true,
    urlcolor = teal,
    linkcolor = white    
}
\usepackage[portuguese]{babel}
\usepackage{subfig}

\newtheorem{obj}{Objetivo}
\newtheorem{ementa}{Ementa}

\begin{document}

\begin{frame}[plain]
    \titlepage{
        \begin{center}
            \begin{minipage}{0.8\textwidth}
                \centering
            \end{minipage}
        \end{center}}
\end{frame}

\section{Docente}
\begin{frame}{Docente}
    \begin{tabular}{cl}
        \begin{tabular}{c}
            \includegraphics[width=3.5cm]{./figures/Paulo}
        \end{tabular}
         & \begin{tabular}{l}
               \parbox{0.6\linewidth}{%  change the parbox width as appropiate
                   \begin{itemize}
                    \item \textbf{Nome:} Paulo Victor da Fonseca
                    \item \textbf{Formação:} Doutorado em Economia - UFSC
                    \item \textbf{Áreas de pesquisa:} Macroeconomia. Políticas monetária e fiscal. Modelos DSGE. Modelos novo-Keynesianos com agentes heterogêneos. Modelos baseados em agentes.
                    \item \textbf{Website:} \href{https://sites.google.com/view/paulovfonseca}{sites.google.com/view/paulovfonseca}

                    \item \textbf{Contato:} \href{mailto:paulo.fonseca@udesc.br}{paulo.fonseca@udesc.br}
                \end{itemize}
               }
           \end{tabular} \\
    \end{tabular}
\end{frame}

\section{Motivação}
\begin{frame}{Métodos Quantitativos em Economia I}
    \begin{itemize}
        \item A ciência econômica tem sido definida como o ``estudo da alocação de recursos escassos''.\bigskip

        \item Em um mundo com recursos escassos e demandas quase ilimitadas, é necessário estabelecer critérios para decidir quais e quantos bens e serviços serão produzidos, e como serão alocados entre os agentes econômicos.\bigskip

        \item É esse mecanismo de \hlight{alocar os recursos escassos aos agentes econ\^{o}micos} que compõe o cerne da ciência econômica.
    \end{itemize}
\end{frame}

\begin{frame}{Métodos Quantitativos em Economia I}
    \begin{itemize}
        \item Sistema econômico: ambiente extremamente complexo -> impossível descrever todas as características que o compõem.\bigskip

        \item Economistas utilizam \tikz[tstyle]{\node[nstyle](node0){modelos econômicos}} para descrever as atividades econômicas.\bigskip
        \begin{tikzpicture}[tpstyle]
            \node[pencil, draw, minimum height=0.7cm, minimum width = 3.5cm] (box0) at (node0) {};
        \end{tikzpicture}

        \item Estes modelos devem abstrair grande parte das complexidades do ``mundo real'' e focar nos elementos \hlight{essenciais} para o objeto de estudo em questão.\bigskip

        \item Apesar de serem abstrações da realidade, modelos fornecem um auxílio fundamental para o entendimento do comportamento econômico.
    \end{itemize}
\end{frame}

\begin{frame}{Métodos Quantitativos em Economia I}
    \begin{itemize}
        \item Modelos econômicos tem por objetivo explicar relações simples e, em sua grande maioria, possuem uma estrutura matemática, evidenciando as relações entre fatores (variáveis) que afetam as decisões dos agentes econômicos e os resultados destas escolhas.\bigskip

        \item O \tikz[tstyle]{\node[nstyle](node0){formalismo}} - uso extensivo da matemática no desenvolvimento de teorias econômicas - tem sido uma característica constante da ciência econômica desde a Revolução Marginalista dos anos 1870s, um processo que tem se intensificado ao longo dos últimos anos.\bigskip
        \begin{tikzpicture}[tpstyle]
            \draw[pencil, very thick] ([yshift=-1pt]node0.south west) to ([yshift=-1pt]node0.south east);
        \end{tikzpicture}

        \item Atualmente, os economistas veem a matemática como uma ferramenta indispensável para qualquer área de estudo, desde as expressões estatísticas de tendências das séries econômicas observáveis ao desenvolvimento de sistemas (modelos) econômicos inteiramente abstratos.
    \end{itemize}
\end{frame}

\begin{frame}{Métodos Quantitativos em Economia I}
    \begin{itemize}
        \item A maioria dos modelos econômicos começa sua estruturação com a hipótese de que agentes econômicos buscam seus objetivos de maneira \hlight{racional}.\medskip

              \begin{itemize}
                  \item Consumidores: \tikz[tstyle]{\node[nstyle](node0){maximizar}} suas utilidades.

                  \item Firmas: \tikz[tstyle]{\node[nstyle](node1){maximizar}} seus lucros (ou \tikz[tstyle]{\node[nstyle](node2){minimizar}} custos).

                  \item Regulamentadores governamentais: \tikz[tstyle]{\node[nstyle](node3){maximizar}} o bem-estar da sociedade.\bigskip
              \end{itemize}

        \begin{tikzpicture}[tpstyle]
            \draw[pencil, very thick, brick] ([yshift=-1pt]node0.south west) to ([yshift=-1pt]node0.south east);
            \draw[pencil, very thick, brick] ([yshift=-1pt]node1.south west) to ([yshift=-1pt]node1.south east);
            \draw[pencil, very thick] ([yshift=-1pt]node2.south west) to ([yshift=-1pt]node2.south east);
            \draw[pencil, very thick, brick] ([yshift=-1pt]node3.south west) to ([yshift=-1pt]node3.south east);
        \end{tikzpicture}

        \item Racionalidade não significa a exclusão de comportamentos prejudiciais ao próprio indivíduo (e.g., fumar ou usar drogas).\bigskip

        \item Os agentes econômicos são racionais no sentido de que suas ações são coerentes com a busca de sua felicidade ou de seus objetivos, mesmo que isso resulte em um comportamento prejudicial à pessoa ou à sociedade.
    \end{itemize}
\end{frame}

\begin{frame}{Métodos Quantitativos em Economia I}
    \begin{itemize}
        \item De outra forma, em economia adotamos a hipótese unificadora de que os agentes econômicos fazem o melhor uso possível dos recursos escassos disponíveis, i.e., \textbf{maximizam} suas funções objetivo sujeito às restrições impostas a eles.\medskip

        \item O critério a ser maximizado e as restrições impostas aos agentes econômicos variam de acordo com os contextos: consumidores - decisões de consumo e oferta de trabalho, firmas - decisões acerca do quanto produzir, governo - políticas econômicas.\medskip

        \item No entanto, todos estes problemas de \hlight{otimiza\c{c}\~{a}o com restri\c{c}\~{a}o} aos quais os agentes econômicos estão condicionados possuem uma estrutura matemática comum o que, por sua vez, resulta em uma intuição econômica comum a todos eles.\medskip

        \item O desenvolvimento e análise desta estrutura formal comum e suas intuições econômicas compõem o escopo do curso de Métodos Quantitativos em Economia I.
    \end{itemize}
\end{frame}

\section{Ementa}
\begin{frame}{Métodos Quantitativos em Economia I: Ementa}
    \begin{center}
        \begin{minipage}{.9\textwidth}
            \NB{\hlight{Otimiza\c{c}\~{a}o irrestrita:} Condições de 1ª e 2ª ordens para máximos e mínimos irrestritos. Aplicações econômicas de otimização irrestrita.\medskip \\
            \hlight{Otimiza\c{c}\~{a}o com restri\c{c}\~{o}es:} Condições de 1ª ordem para otimização condicionada com restrições de igualdade e desigualdade.  Método dos multiplicadores de Lagrange e de Kuhn Tucker. Condições de 2ª ordem para otimização condicionada com restrições de igualdade e desigualdade.  Interpretação dos multiplicadores em problemas de otimização.  Teorema do envelope. \medskip \\
            \hlight{Homogeneidade, homoteticidade, concavidade, quase-concavidade:} Funções homogêneas, homotéticas, côncavas e quase côncavas. \medskip \\
            \hlight{Aplica\c{c}\~{o}es econ\^{o}micas:} Aplicações econômicas dos problemas de otimização relacionados à maximização de utilidade e demanda maximização de lucros, custos, ótimo de Pareto e teoremas fundamentais de bem-estar. \medskip \\
            \hlight{Programa\c{c}\~{a}o linear:} Programação linear.
            }
        \end{minipage}
    \end{center}

\end{frame}

\section{Objetivo}
\begin{frame}{Métodos Quantitativos em Economia I: objetivo}
    \begin{center}
        \begin{minipage}{.9\textwidth}
            \NB{O objetivo da disciplina é apresentar aos alunos as principais técnicas de otimização estática, bem como suas principais aplicações em Economia. \\
                Ao final do curso espera-se que o aluno seja capaz de utilizar o ferramental desenvolvido na disciplina em aplicações à Teoria Econômica (microeconomia, macroeconomia e disciplinas correlatas).}
        \end{minipage}
    \end{center}

    O curso será dividido em seis blocos:
    \begin{enumerate}
        \item Introdução e revisão de conceitos básicos

        \item Otimização estática sem restrições

        \item Otimização estática com restrições

        \item Funções homogêneas e funções homotéticas

        \item Concavidade e quase-concavidade

        \item Programação linear
    \end{enumerate}
\end{frame}

\section{Formato das aulas e avaliações}
\begin{frame}{Formato das aulas e sistema de avaliação}
    \begin{itemize}
        \item A disciplina apoia-se, fundamentalmente, em livros-texto e notas de aula e será ministrada por meio de aulas expositivas.\bigskip

        \item As aulas acontecerão às:
              \begin{itemize}
                  \item Quartas-feiras das 10:15 às 11:55
                  \item Sextas-feiras das 10:15 às 11:55\bigskip
              \end{itemize}

        \item A avaliação será realizada a partir dos procedimentos abaixo:
              \begin{itemize}
                  \item Atividade avaliativa I (PI): 30\%
                  \item Atividade avaliativa II (PII): 30\%
                  \item Atividade avaliativa III (PIII): 20\%
                  \item Trabalhos adicionais: 20\%\bigskip
              \end{itemize}

        \item Página da disciplina no GitHub: \href{github.com/pvfonseca/MetodosQuantitativos}{https://github.com/pvfonseca/MetodosQuantitativos}
    \end{itemize}
\end{frame}

\begin{frame}{Formato das aulas e sistema de avaliação}
    \begin{itemize}
        \item Os alunos devem ter em mente que o aprendizado e o acompanhamento do curso dependem essencialmente de seu próprio esforço.\bigskip

        \item Os tópicos do programa serão apresentados em aulas expositivas, destinadas à apresentação de conceitos, modelos e suas aplicações.\bigskip

        \item \hlight{Embora importantes, as aulas n\~{a}o podem jamais ser vistas como substitutas da leitura regular e cuidadosa dos textos indicados e da resolu\c{c}\~{a}o dos exerc\'{i}cios propostos.}
    \end{itemize}

\end{frame}
\section{Bibliografia}

\begin{frame}{Bibliografia}
    \begin{figure}
        \centering
        \subfloat[Chiang e Wainwright (2006)\label{fig1a}]{\includegraphics[width=0.2\textwidth]{./figures/chiang}} \quad
        \subfloat[Simon e Blume (2004)\label{fig1b}]{\includegraphics[width=0.2\textwidth]{./figures/simon}} \quad
        \subfloat[Hoy et al. (2022)\label{fig1c}]{\includegraphics[width=0.25\textwidth]{./figures/hoy}} \quad
        \subfloat[Nicholson e Snyder (2019)\label{fig1d}]{\includegraphics[width=0.2\textwidth]{./figures/nicholson}}
        \caption{Bibliografia do curso}
        \label{fig1}
    \end{figure}
\end{frame}

\begin{frame}{Bibliografia}
    \begin{figure}
        \centering
        \subfloat[Dixit (1990)\label{fig3a}]{\includegraphics[width=0.25\textwidth]{./figures/dixit}} \quad
        \subfloat[Fuente (2000)\label{fig3b}]{\includegraphics[width=0.25\textwidth]{./figures/fuente}} \quad
        \subfloat[Sydsaeter et al. (2016)\label{fig3c}]{\includegraphics[width=0.25\textwidth]{./figures/carvajal}}
        \caption{Bibliografia do curso}
        \label{fig3}
    \end{figure}
\end{frame}

\begin{frame}{Bibliografia}
    \begin{figure}
        \centering
        \subfloat[Silberberg e Suen (2001)\label{fig2b}]{\includegraphics[width=0.25\textwidth]{./figures/silberberg}} \quad
        \subfloat[Stewart (2017)\label{fig2c}]{\includegraphics[width=0.25\textwidth]{./figures/stewart1}} \quad
        \subfloat[Stewart (2017)\label{fig2d}]{\includegraphics[width=0.25\textwidth]{./figures/stewart2}}
        \caption{Bibliografia do curso}
        \label{fig2}
    \end{figure}
\end{frame}

\begin{frame}{Bibliografia}
    \begin{itemize}
        \item CHIANG, A.C.; WAINWRIGHT, K. \emph{Matemática para economistas}. Rio de Janeiro: Elsevier, 2006.
        \item DIXIT, A. \emph{Optimization in Economic Theory}. 2.ed., Oxford University Press, 1990.
        \item HOY, M.; LIVERNOIS, J.; McKENNA, C.; REES, R.; STENGOS, T. \emph{Mathematics for Economics}. 2.ed., Massachusetts: MIT Press, 2001.
        \item FUENTE, A. \emph{Mathematical methods and models for economists}. Cambridge, UK. New York, NY: Cambridge University Press, 2000.
        \item NICHOLSON, W.; SNYDER C. \emph{Teoria microeconômica: Princípios básicos e aplicações}. Cengage Learning Brasil, 2019. Disponível em: \href{https://app.minhabiblioteca.com.br/books/9788522127030/}{app.minhabiblioteca.com.br/books/9788522127030}
        \item SILBERBERG, E.; SUEN, W. \emph{The structure of economics: a mathematical analysis}. 3rd.ed. Singapore: McGraw-Hill Higher Education, 2001.
        \item SIMON, C.P.; BLUME, L. \emph{Matemática para economistas}. Porto Alegre: Bookman, 2004.
        \item STEWART, J. \emph{Cálculo – Volume 1}. 8.ed. Cengage Learning Brasil, 2017. Disponível em: \href{https://app.minhabiblioteca.com.br/books/9788522126859/}{app.minhabiblioteca.com.br/books/9788522126859}
        \item STEWART, J. \emph{Cálculo – Volume 2}. 8.ed. Cengage Learning Brasil, 2017. Disponível em: \href{https://app.minhabiblioteca.com.br/books/9788522126866/}{app.minhabiblioteca.com.br/books/9788522126866}
        \item SYDSÆTER, K.; HAMMOND, P.J.; STRØM, A.; CARVAJAL, A. \emph{Essential mathematics for economic analysis}. 5th.ed. Harlow, UK: Pearson Education Limited, 2016.
    \end{itemize}
\end{frame}
\end{document}