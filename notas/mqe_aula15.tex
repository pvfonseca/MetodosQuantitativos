\documentclass[10pt]{beamer}
\usetheme{jambro}

\title[]{Programação Linear}
\author[]{\href{https://pvfonseca.github.io/}{Paulo Victor da Fonseca}}
\date{}

\hypersetup{
    colorlinks = true,
    urlcolor = teal,
    linkcolor = teal    
}
\usepackage[portuguese]{babel}
\usepackage{subfig}
\usepackage{emoji}
\usepackage{hyperref}
\newtheorem{obj}{Objetivo}
\newtheorem{teo}{Teorema}
\newtheorem{defi}{Definição}

\begin{document}

\begin{frame}[plain]
    \titlepage{
        \begin{center}
            \begin{minipage}{0.8\textwidth}
                \centering
            \end{minipage}
        \end{center}}
\end{frame}

\begin{frame}{Sumário}
    \tableofcontents
\end{frame}

\section{Introdução}
\begin{frame}{Introdução}
    \begin{itemize}
        \item \hlight{Programação Linear (LP)}: problemas de otimização condicionados nos quais o objetivo é maximizar ou minimizar uma função linear sujeito a um conjunto de restrições de igualdade e/ou desigualdade lineares\bigskip
        \item Em princípio qualquer problema LP pode ser resolvido numericamente, desde que uma solução exista: método simplex fornece um algoritmo numérico eficiente que encontra o conjunto solução em um número finito de passos\bigskip
        \item Não discutiremos método simplex, discutiremos a teoria básica de problemas LP ao invés de detalhes do método simplex\bigskip
        \item A partir de desenvolvimentos nos anos 80, o método simplex deixou de ser o estado da arte para solucionar grandes problemas LP numericamente
    \end{itemize}
\end{frame}

\section{Abordagem gráfica}
\begin{frame}{Abordagem gráfica}
    \begin{itemize}
        \item Um problema LP geral em apenas duas variáveis envolve maximizar ou minimizar uma função objetivo linear:
        \begin{equation*}
            z = c_1x_1 + c_2x_2
        \end{equation*}
        sujeito a $m$ restrições de desigualdade lineares:
        \begin{align*}
            a_{11}x_1 + a_{12}x_2 &\leq b_1\\
            a_{21}x_1 + a_{22}x_2 &\leq b_2\\
             &\ddots \\
            a_{m1}x_1 + a_{m2}x_2 &\leq b_m
        \end{align*}
        \item Normalmente impomos, também, restrições de não-negatividade:
        \[
          x_1\geq 0, \qquad x_2\geq 0  
        \]
    \end{itemize}
\end{frame}

\begin{frame}{Abordagem gráfica}
    \begin{itemize}
        \item \hlight{Problema de produção (Bertsimas \& Tsitsiklis, 1997):} uma fábrica produz dois produtos - Produto 1 e Produto 2\bigskip
        \item A produção de cada bem requer tanto matéria-prima quanto mão-de-obra\bigskip
        \item A venda de cada produto gera uma receita\bigskip        
    \end{itemize}
    \begin{table}[h]
        \centering
        \begin{tabular}{l|r|r}
        \hline
         & \textbf{Produto 1} & \textbf{Produto 2} \\
        \hline
        \hline
        \textbf{Material} & 2 & 5 \\
        \textbf{Trabalho} & 4 & 2 \\
        \textbf{Receita} & 3 & 4 \\
        \hline
        \end{tabular}
    \end{table}
    \begin{itemize}
        \item 30 unidades de material e 20 unidades de trabalho estão disponíveis\bigskip
        \item Objetivo é construir um plano de produção que maximize receita
    \end{itemize}    
\end{frame}

\begin{frame}{Aborgadem gráfica}
    \begin{itemize}
        \item Formulação do problema:
        \begin{align*}
            \max_{x_1,x_2} \quad & z = 3x_1 + 4x_2\\
            \text{s.a.} \quad & 2x_1 + 5x_2 \leq 30\\
            & 4x_1 + 2x_2 \leq 20\\
            & x_1, x_2 \geq 0
        \end{align*}
    \end{itemize}
\end{frame}

\begin{frame}{Aborgadem gráfica}
    \begin{figure}
        \centering
        \includegraphics[width=0.6\textwidth]{./figures/aula15_fig1.png}
        \caption{Solução gráfica do problema de produção}        
    \end{figure}
\end{frame}

\begin{frame}{Abordagem gráfica}
    \begin{itemize}
        \item \hlight{Nomenclatura}\bigskip
        \item \hlight{Solução viável:} um vetor $x$ que satisfaz todas as restrições\bigskip
        \item \hlight{Conjunto factível:} o conjunto de todas as soluções viáveis\bigskip
        \item \hlight{Solução ótima:} uma solução viável $x$ que maximiza ou minimiza a função objetivo\bigskip
        \item \hlight{Valor ótimo:} o valor da função objetivo na solução ótima\bigskip
        \item Se o conjunto factível é vazio, dizemos que o problema LP é não-factível
    \end{itemize}    
\end{frame}

\begin{frame}{Abordagem gráfica}
    \begin{itemize}
        \item Na Figura, a região azul é o conjunto factível no qual todas as restrições são satisfeitas\bigskip
        \item As linhas paralelas laranjas são as curvas de nível da função objetivo (iso-receitas)\bigskip
        \item O objetivo da firma é encontrar as linhas laranjas paralelas ao limite superior do conjunto factível\bigskip
        \item A intersecção do conjunto factível e a linha laranja mais alta delimita o conjunto factível\bigskip
        \item Neste caso, o conjunto factível é o ponto $\left(\frac{5}{2}, 5\right)$
    \end{itemize}
\end{frame}

\begin{frame}{Exemplo 2}    
    \begin{itemize}
            \item Um padeiro tem 150 kg de farinha, 22kg de açúcar e 27,5kg de manteiga com os quais prepara 2 tipos de bolo\bigskip
            \item Suponha que a produção de 1 dúzia de bolos do tipo A requer 3kg de farinha, 1kg de açúcar e 1kg de manteiga\bigskip
            \item Suponha que a produção de 1 dúzia de bolos do tipo B requer 6kg de farinha, 1/2 kg de açúcar e 1kg de manteiga\bigskip
            \item A receita de venda de 1 dúzia de bolos do tipo A é de 20 e a receita de venda de 1 dúzia de bolos do tipo B é de 30\bigskip
            \item Quantas dúzias de bolos do tipo A ($x_1$) e quantas dúzias de bolos do tipo B ($x_2$) o padeiro deve produzir para maximizar sua receita?
    \end{itemize}
\end{frame}

\begin{frame}{Exemplo 2}
    \begin{itemize}
        \item Formulação do problema:
        \begin{align*}
            \max_{x_1,x_2} \quad & z = 20x_1 + 30x_2\\
            \text{s.a.} \quad & 3x_1 + 6x_2 \leq 150\\
            & x_1 + 0,5x_2 \leq 22\\
            & x_1 + x_2 \leq 27,5 \\
            & x_1, x_2 \geq 0
        \end{align*}
    \end{itemize}
\end{frame}

\begin{frame}{Exemplo 2}
    \begin{figure}
        \centering
        \includegraphics[width=\textwidth]{./figures/aula15_fig2.png}        
    \end{figure}
\end{frame}

\begin{frame}{Exemplo 3}
    \begin{itemize}
        \item Uma firma produz dois bens, A e B\bigskip
        \item A firma possui 2 fábricas que, conjuntamente, produzem os 2 bens nas seguintes quantidades (por hora):
    \end{itemize}
    \begin{table}[h]
        \centering
        \begin{tabular}{l|r|r}
        \hline
         & \textbf{Fábrica 1} & \textbf{Fábrica 2} \\
        \hline
        \hline
        \textbf{Bem A} & 10 & 20 \\
        \textbf{Bem B} & 25 & 25 \\
        \hline
        \end{tabular}
    \end{table}
    \begin{itemize}
        \item A firma recebe uma encomenda de 300 unidades do Bem A e 500 unidades do Bem B\bigskip
        \item Os custos de operação das fábricas são de 10000 e 8000 por hora, respectivamente
    \end{itemize}
\end{frame}

\begin{frame}{Exemplo 3}
    \begin{itemize}
        \item Formulação do problema:
        \begin{align*}
            \min_{u_1,u_2} \quad & z = 10000u_1 + 8000u_2\\
            \text{s.a.} \quad & 10u_1 + 20u_2 \geq 300\\
            & 25u_1 + 25u_2 \geq 500\\            
            & u_1, u_2 \geq 0
        \end{align*}
    \end{itemize}
\end{frame}

\begin{frame}{Exemplo 3}
    \begin{figure}
        \centering
        \includegraphics[width=0.8\textwidth]{./figures/aula15_fig3.png}        
    \end{figure}
\end{frame}

\section{Problema LP generalizado}
\begin{frame}
    \begin{itemize}
        \item Abordagem gráfica funciona bem quando temos apenas duas variáveis de decisão\bigskip
        \item O método pode ser estendido para 3 variáveis de decisão: o conjunto factível, neste caso, é um poliedro convexo no espaço tridimensional e as superfícies de nível da função objetivo são planos no espaço tridimensional\bigskip
        \item Não é fácil visualizar a solução nestes casos\bigskip
        \item Para mais que três variáveis de decisão, não temos abordagens gráficas disponíveis\bigskip
        \item No entanto, podemos usar \hlight{teoria da dualidade} para resolver problemas LP graficamente quando ou o número de incógnitas ou o número de restrições é menor ou igual a 3
    \end{itemize}
\end{frame}

\begin{frame}{Problema LP generalizado}
    \begin{itemize}
        \item Objetivo é maximizar ou minimizar a função objetivo:
        \begin{equation}
            z = c_1x_1 + c_2x_2 + \cdots + c_nx_n = c'x
        \end{equation}
        sujeito a $m$ restrições lineares de desigualdade:
        \begin{eqnarray}
            a_{11}x_1 + a_{12}x_2 + \cdots + a_{1n}x_n &\leq& b_1\nonumber\\
            a_{21}x_1 + a_{22}x_2 + \cdots + a_{2n}x_n &\leq& b_2\nonumber\\
            \vdots &\ddots& \vdots\\
            a_{m1}x_1 + a_{m2}x_2 + \cdots + a_{mn}x_n &\leq& b_m\nonumber
        \end{eqnarray}
        usualmente impomos restrições de não-negatividade:
        \begin{equation}
            x_1\geq 0, \quad x_2\geq 0, \quad \cdots, \quad x_n\geq 0
        \end{equation}        
    \end{itemize}
\end{frame}

\begin{frame}{Problema LP generalizado}
    \begin{itemize}
        \item O conjunto de soluções viáveis, neste caso, é um poliedro convexo no ortante não-negativo do espaço $n$-dimensional\bigskip
        \item Para o caso tridimensional temos o seguinte exemplo:
    \end{itemize}
    \begin{figure}
        \centering
        \includegraphics[width=0.3\textwidth]{./figures/aula15_fig4.png}        
        \caption{Um poliedro convexo no ortante não-negativo do espaço tridimensional}
    \end{figure}
\end{frame}

\begin{frame}{Problema LP generalizado}
    \begin{itemize}
        \item Os pontos O, P, Q, R, S, T, U, V são pontos extremos\bigskip
        \item Os segmentos de reta OP, OT, OV, etc. são arestas\bigskip
        \item As porções planas da fronteira que são triângulos ou quadriláteros são faces\bigskip
        \item Estes objetos também estão presentes em qualquer poliedro convexo do espaço $n$-dimensional\bigskip
        \item Se $n$ e $m$ são grandes, o número de pontos extremos pode ser gigantesco, mas o método simplex pode resolver este tipo de problemas
    \end{itemize}    
\end{frame}

\section{Teoria da dualidade: introdução}
\begin{frame}{Problema dual}
    \begin{itemize}
        \item O que acontece com a solução ótima se a disponibilidaed de recursos é alterada?\bigskip
        \item Para problemas de programação linear, as respostas a questões deste tipo estão relacionadas à teoria da dualidade de LP\bigskip
        \item Considere, novamente, o problema do padeiro [NA LOUSA]
    \end{itemize}
\end{frame}

\begin{frame}{Problema dual}
    \begin{itemize}
        \item Considere o problema LP geral:
        \begin{align}
            \max \quad & c_1 x_1 + \dots + c_n x_n \\
            \text{s.a.} \quad & a_{11}x_1 + \dots + a_{1n}x_n \leq b_1\nonumber\\
            & \qquad \vdots \nonumber\\
            & a_{m1}x_1 + \dots + a_{mn}x_n \leq b_m\nonumber\\
            & x_1 \geq 0, \dots, x_n \geq 0\nonumber
        \end{align}
        \item O problema dual associado é dado por:
        \begin{align}
            \min \quad & b_1 u_1 + \dots + b_m u_m \\
            \text{s.a.} \quad & a_{11}u_1 + \dots + a_{m1}u_m \geq c_1\nonumber\\
            & \qquad \vdots \nonumber\\
            & a_{1n}u_1 + \dots + a_{mn}u_m \geq c_n\nonumber\\
            & u_1 \geq 0, \dots, u_m \geq 0\nonumber
        \end{align}
    \end{itemize}
\end{frame}

\begin{frame}{Problema dual}
    \begin{itemize}
        \item Em formulação matricial, o problema primal é dado por:
        \begin{equation}
            \max \mathbf{c'x} \quad \text{s.a.} \quad \mathbf{Ax \leq b}, \quad \mathbf{x\geq 0}
        \end{equation}
        \item Em formulação matricial, o problema dual é dado por:
        \begin{equation}
            \min \mathbf{b'u} \quad \text{s.a.} \quad \mathbf{A'u \geq c}, \quad \mathbf{u\geq 0}
        \end{equation}
        \item Ou, de forma mais conveniente:
        \begin{equation}
            \min \mathbf{u'b} \quad \text{s.a.} \quad \mathbf{u'A \geq c'}, \quad \mathbf{u'\geq 0}
        \end{equation}
    \end{itemize}
\end{frame}

\section{Teorema da dualidade}
\begin{frame}{Teorema da dualidade}
    \begin{teo}[1]
        Se $(x_1, \dots, x_n)$ é uma solução viável para o problema primal e $(u_1, \dots, u_m)$ é uma solução viável para o problema dual, então:
        \begin{equation}
            b_1 u_1 + \dots + b_m u_m \geq c_1 x_1 + \dots + c_n x_n
        \end{equation}
        De forma que a função objetivo no problema dual sempre tem um valor pelo menos tão grande quanto do problema primal.
    \end{teo}
\end{frame}

\begin{frame}{Teorema da dualidade}
    \begin{teo}[2]
        Suponha que $(x_1^*, \dots, x_n^*)$ e $(u_1^*, \dots, u_m^*)$ são soluções viáveis para o problema primal e dual, respectivamente, e que:
        \begin{equation}
            b_1 u_1^* + \dots + b_m u_m^* = c_1 x_1^* + \dots + c_n x_n^*
        \end{equation}
        Então $(x_1^*, \dots, x_n^*)$ é uma solução do problema primal e $(u_1^*, \dots, u_m^*)$ uma solução para o problema dual.
    \end{teo}
\end{frame}

\begin{frame}{Teorema da dualidade}
    \begin{teo}[Teorema da Dualidade]
        Suponha que o problema primal possui uma solução ótima finita. Então, o problema dual também possui uma solução ótima finita e os valores correspondentes das funções objetivo são iguais.
        
        \bigskip
        Se o problema primal não possui um ótimo limitado, então, o problema dual não possui solução viável.
        
        \bigskip
        De forma simétrica, se o problema primal não possui solução viável, então, o problema dual não possui um ótimo limitado.
    \end{teo}
\end{frame}

\section{Bibliografia}
\begin{frame}{\emoji{books} Bibliografia}
    \begin{itemize}                
        \item CHIANG, A.C.; WAINWRIGHT, K. Matemática para economistas. Rio de Janeiro: Elsevier, 2006\medskip
        \item DE LA FUENTE, Á. Mathematical methods and models for economists. Cambridge University Press, Cambridge, UK, 2000\medskip
        \item NICHOLSON, W.; SNYDER C. Teoria microeconômica: Princípios básicos e aplicações. Cengage Learning Brasil, 2019. Disponível em: \href{https://app.minhabiblioteca.com.br/books/9788522127030/}{app.minhabiblioteca.com.br/books/9788522127030/}\medskip
        \item SIMON, C.P.; BLUME, L. Matemática para economistas. Porto Alegre: Bookman, 2004\medskip
        \item SYDSÆTER, K.; HAMMOND, P.J.; STRØM, A.; CARVAJAL, A. Essential mathematics for economic analysis. 5th.ed. Harlow, UK: Pearson Education Limited, 2016
    \end{itemize}
\end{frame}
\end{document}