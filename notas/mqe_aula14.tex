\documentclass[10pt]{beamer}
\usetheme{jambro}

\title[]{Funções homogêneas e funções homotéticas}
\author[]{\href{https://pvfonseca.github.io/}{Paulo Victor da Fonseca}}
\date{}

\hypersetup{
    colorlinks = true,
    urlcolor = teal,
    linkcolor = teal    
}
\usepackage[portuguese]{babel}
\usepackage{subfig}
\usepackage{emoji}
\usepackage{hyperref}
\newtheorem{obj}{Objetivo}
\newtheorem{teo}{Teorema}
\newtheorem{defi}{Definição}

\begin{document}

\begin{frame}[plain]
    \titlepage{
        \begin{center}
            \begin{minipage}{0.8\textwidth}
                \centering
            \end{minipage}
        \end{center}}
\end{frame}

\begin{frame}{Sumário}
    \tableofcontents
\end{frame}

\section{Introdução}
\subsection{Introdução}
\begin{frame}{Introdução}
    \begin{itemize}
        \item Para estudarmos de maneira eficiente a estrutura de muitos dos modelos econômicos é necessário compreendermos uma importante classe de funções conhecida como \textcolor{purple}{funções homogêneas}.
         \bigskip
        \item O interesse nessas funções emergiu de um problema na teoria econômica da distribuição.
        \bigskip
        \item O desenvolvimento da teoria da produtividade marginal por Alfred Marshall (revolução marginalista e escola neoclássica), entre outros, levou à conclusão de que os fatores de produção seriam remunerados de acordo com seus produtos marginais.
        \bigskip
        \item Dito de outra forma, os fatores seriam empregados até o ponto em que sua contribuição para à produção da firma seja exatamente igual ao custo de aquisição de unidades adicionais deste fator.
    \end{itemize}
\end{frame}

\begin{frame}{Introdução}
    \begin{figure}
        \centering
        \includegraphics[width=0.3\textwidth]{./figures/aula14_fig1.jpg}
        \caption{Alfred Marshall (1842-1924). Fonte: \href{https://en.wikipedia.org/wiki/Alfred_Marshall}{Wikipedia}.}
        \label{a.marshall}
    \end{figure}
\end{frame}

\begin{frame}{Introdução}
    \begin{itemize}
        \item Seja $y = f(x_1, x_2)$ a função de produção de uma firma.
        \bigskip
        \item Além disso, $w_i$ denota a remuneração do fator $x_i$ e $p$ o preço do produto desta firma, a regra desenvolvida é, portanto:
        \[
            p \text{PMg}_i = pf_i = w_i, \qquad f_i \equiv \frac{\partial f}{\partial x_i}.    
        \]
         
        \item No entanto, esta análise foi desenvolvida em um arcabouço de \textcolor{purple}{equilíbrio parcial}. I.e., cada fator era analisado de forma independente.
        \bigskip
        \item Uma questão naturalmente emerge: como é possível assegurar que a firma é capaz de fazer estes pagamentos para ambos fatores?
        \bigskip
        \item Todas as remunerações dos fatores devem derivar da produção da firma.
        \bigskip
        \item Seria produzido uma quantidade suficiente (ou haveria uma superprodução?) para remunerar cada fator pelo seu produto marginal?
    \end{itemize}
\end{frame}

\begin{frame}{Introdução}
    \begin{itemize}
        \item Um teorema desenvolvido pelo matemático suíço Euler ajudou a resolver este problema.
        \bigskip
        \item Veremos que se a função de produção exibe retornos constantes de escala, a soma da remuneração total de fatores será idêntica à produção total da firma.
    \end{itemize}
\end{frame}

\begin{frame}{Introdução}
    \begin{itemize}
        \item Economistas normalmente trabalham com funções que possuem algumas propriedades fortes tais como homogeneidade ou convexidade.
        \bigskip
        \item Em algumas vezes, essas propriedades emergem naturalmente para funções específicas.
        \bigskip
        \item Por exemplo, funções de demanda são naturalmente homogêneas nos preços e na renda.
        \bigskip
        \item Outras, no entanto, os economistas impõe estas propriedades como hipóteses para demonstrar teoremas sobre modelos econômicos.
        \bigskip
        \item Por exemplo, conseguimos derivar bem mais resultados para modelos econômicos quando as utilidades são funções homotéticas ou quando as funções lucro são côncavas do que conseguiríamos sem estas hipóteses.
    \end{itemize}
\end{frame}

\section{Funções homogêneas}
\subsection{Definição}
\begin{frame}{Funções homogêneas}
    \begin{itemize}
        \item Considere as seguintes funções:
        \begin{eqnarray*}
            f(x,y) &=& x^2 + y^2, \\
            g(x,y) &=& \frac{xy}{x + y}, \\
            h(x,y) &=& x^2y\ln \frac{y}{x}.
        \end{eqnarray*}
         
        \item Cada uma dessas funções apresenta uma propriedade interessante de que, se as variáveis $x, y$ forem multiplicadas por um parâmetro $t$ qualquer, então, nós obtemos a função original multiplicada por uma potência de $t$.
        \bigskip
        \item Funções que apresentam essa propriedade são chamadas \textcolor{blue}{funções homogêneas}.
    \end{itemize}
\end{frame}

\begin{frame}{Funções homogêneas}
    \begin{itemize}
        \item Funções homogêneas emergem naturalmente em várias áreas da economia.
        \bigskip
        \item Funções lucro e funções custo, que são derivadas das funções de produção, e funções de demanda, que são derivadas de funções utilidade, são automaticamente homogêneas nos modelos econômicos convencionais.
    \end{itemize}
\end{frame}

\begin{frame}{Funções homogêneas}
    \begin{defi}[Funções homogêneas]
        Para qualquer escalar $k$, uma função de valores reais $f(x_1, \dots, x_n)$ é \textcolor{blue}{homogênea de grau $k$} se:
        \begin{equation}
            f(tx_1, \dots, tx_n) = t^k f(x_1, \dots, x_n),
            \label{eq1}
        \end{equation}
        para todos os valores de $x_1, \dots, x_n$ e $t$ para os quais as funções $f(x_1, \dots, x_n)$ e  $f(tx_1, \dots, tx_n)$ estão definidas.
    \end{defi}
\end{frame}

\begin{frame}{Funções homogêneas}
    \begin{itemize}
        \item O grau $k$ em nossa definição é uma constante. Não precisa, necessariamente, ser um número inteiro. Além disso, pode ser um número negativo ou igual a zero.
        \bigskip
        \item Se nossa relação fundamental (\ref{eq1}) é válida apenas para valores de $t$ restritos ao ortante positivo, $\mathbb{R}_{++}^n$ (ou não-negativo), dizemos que $f$ é \textbf{positivamente homogênea de grau $k$}.
        \bigskip
        \item Como em economia normalmente trabalharemos com funções homogêneas definidas no ortante positivo ($t > 0$), por conveniência utilizaremos apenas o termo ``função homogênea de grau $k$''.
    \end{itemize}
\end{frame}

\subsection{Exemplos}
\begin{frame}{Exemplos}
    \begin{enumerate}
        \item $\frac{x}{x^2 + y^2}$ é homogênea de grau -1.
        \bigskip
        \item $x^{1/3} + xy^{-2/3}$ é homogênea de grau $1/3$.
        \bigskip
        \item $\frac{x^2 - y^2}{x^2 + y^2}$ é homogênea de grau 0.
    \end{enumerate}
\end{frame}

\subsection{Funções homogêneas em economia}
\begin{frame}{Funções de produção}
    \begin{itemize}
        \item Em economia, normalmente é conveniente trabalhar com funções de produção que sejam funções homogêneas.
        \bigskip
        \item Seja $f(x_1, \dots, x_n)$ uma função de produção homogênea de grau $k$, temos que:
        \[
        f(tx_1, \dots, tx_n) = t^kf(x_1, \dots, x_n).
        \]
         
        \item Se $k = 1$, dizemos que a firma apresenta \textbf{retornos constantes de escala}.
        \bigskip
        \item Se $k > 1$, dizemos que a firma apresenta \textbf{retornos crescentes de escala}.
        \bigskip
        \item Se $k < 1$, dizemos que a firma apresenta \textbf{retornos decrescentes de escala}.
    \end{itemize}
\end{frame}

\begin{frame}{Funções de produção}
    \begin{itemize}
        \item Uma forma funcional específica de função homogênea que frequentemente é utilizada em economia é a função de produção do tipo Cobb-Douglas:
        \[
        q = Ax_1^{a_1} \dots x_n^{a_n}, \qquad A, a_1, \dots, a_n > 0.
        \]
         
        \item A função de produção do tipo Cobb-Douglas é homogênea de grau $k = a_1 + \dots + a_n$.
        \bigskip
        \item Portanto, a função de produção Cobb-Douglas exibirá retornos decrescentes, constantes ou crescentes de escala a depender da soma de seus expoentes.
    \end{itemize}
\end{frame}

\begin{frame}{Funções de produção}
    \begin{itemize}
       \item Desde a publicação do trabalho do matemático Charles Cobb e do economista Paul Douglas nos anos 1920, economistas interessados em estimar funções de produção para uma firma ou indústria específica, normalmente, tentam encontrar a função de produção Cobb-Douglas que melhor se ajusta aos dados de insumo-produto da firma.
       \bigskip
       \item Para tanto, podemos estimar uma regressão linear por MQO tomando o logaritmo natural da função de produção:
       \[
       \ln q = \ln A + a_1 \ln x_1 + \dots + a_n \ln x_n.
       \]
    \end{itemize}
\end{frame}

\begin{frame}{Funções de produção}
    \begin{itemize}
        \item Mostre que a função de produção de elasticidade de substituição constante (CES):
        \[
            g(\mathbf{x}) = A\left(\sum_{i=1}^n \delta_i x_i^{-\rho}\right)^{-\nu/\rho},
        \]
        é homogênea de grau $\nu$.
        \bigskip
        \item Restrições sobre os parâmetros: $A>0, \nu>0, \rho>-1, \rho \neq 0, \delta_i>0$, $\forall i$, $\sum_{i=1}^n \delta_i = 1$.
    \end{itemize}
\end{frame}

\begin{frame}{Funções de demanda}
    \begin{itemize}
        \item Enquanto as funções de produção são funções homogêneas por \textcolor{blue}{hipótese}, as funções de demanda são homogêneas por \textcolor{blue}{natureza} (pelo menos se ignorarmos o fenômeno de ilusão monetária).
        \bigskip
        \item Lembre-se que uma função de demanda é a solução do problema primal de maximização de utilidade de um consumidor, ou seja:
        \[
        x = x^d(p_1, \dots, p_n, I) = \max U(x_1, \dots, x_n) \quad \text{s.r.} \quad p_1 x_1 + \dots p_n x_n \leq I.
        \]
         
        \item Note que se multiplicarmos todos os preços e a renda deste consumidor por uma variável positiva $t$ qualquer, sua restrição orçamentária não será alterada.
        \bigskip
        \item Em particular, a cesta de consumo ótima $x$ não seria afetada. Em termos de função demanda:
        \[
        x^d(tp_1, \dots, tp_n, tI) = x^d(p_1, \dots, p_n, I).
        \]
    \end{itemize}
\end{frame}

\subsection{Propriedades de funções homogêneas}
\begin{frame}{Propriedades de funções homogêneas}
    \begin{teo}
        Seja $f(x_1, \dots, x_n)$ uma função homogênea de grau $k$, então, suas derivadas parciais de primeira ordem serão homogêneas de grau $k - 1$.
    \end{teo}
\end{frame}

\begin{frame}{Propriedades de funções homogêneas}
    \begin{teo}[Teorema de Euler]
    Seja $f(x_1, \dots, x_n)$ uma função $C^1$ homogênea de grau $k$ no $\mathbb{R}_+^n$. Então, para quaisquer $x_1, \dots, x_n$:
    \begin{equation}
        x_1 \frac{\partial f(x_1, \dots, x_n)}{\partial x_1} + \dots + x_n \frac{\partial f(x_1, \dots, x_n)}{\partial x_n} = kf(x_1, \dots, x_n),
        \label{eq2}
    \end{equation}
    ou, em notação de gradiente:
    \begin{equation*}
        \mathbf{x} \nabla f(\mathbf{x}) = f(\mathbf{x}).
    \end{equation*}
    \end{teo}
\end{frame}

\begin{frame}[t]{Exercícios}
    \begin{itemize}
        \item Dizemos que uma função é linearmente homogênea se for de grau um.
        \bigskip
        \begin{enumerate}
            \item Dada a função de produção linearmente homogênea $Q = f(K,L)$, o produto médio do trabalho e do capital podem ser expressos como funções da razão capital trabalho apenas.
            \bigskip
            \item Dada a função de produção linearmente homogênea $Q = f(K,L)$, os produtos físicos marginais do capital e do trabalho podem ser expressos como funções apenas de $k = \frac{K}{L}$. Mais precisamente, $PMg_K = f'(k)$ e $PMg_L = f(k) - kf'(k)$.
        \end{enumerate}
    \end{itemize}
\end{frame}

\begin{frame}{Aplicação econômica do teorema de Euler}
    \begin{itemize}
        \item Uma aplicação convencional do teorema de Euler em economia é o de exaustão do produto total de firmas com funções de produção homogêneas.
        \bigskip
        \item Se a função de produção de uma firma é homogênea de grau um, então, pelo teorema de Euler:
        \[
        x_1 \frac{\partial f(\mathbf{x})}{\partial x_1} + \dots + x_n \frac{\partial f(\mathbf{x})}{\partial x_n} = f(\mathbf{x}) = q.
        \]
         
        \item Suponha o critério usual de maximização de lucros de que a firma paga a cada fator de produção $x_i$ a receita de seu produto marginal $p (\partial f/\partial x_i)$, de modo que ela contrata cada fator até o ponto em que sua contribuição para o produto da firma seja igual ao custo de obter uma unidade adicional deste fator.
    \end{itemize}
\end{frame}

\begin{frame}{Aplicação econômica do teorema de Euler}
    \begin{itemize}
        \item Então, o pagamento total da firma será:
        \[
        x_1 p \frac{\partial f(\mathbf{x})}{\partial x_1} + \dots + x_n p \frac{\partial f(\mathbf{x})}{\partial x_n}.
        \]
         
        \item Mas, pelo teorema de Euler, essa expressão é igual a $pq$, ou seja, o produto total da firma.
        \bigskip
        \item Portanto, a receita de uma firma com retornos constantes de escala é exatamente exaurida com os pagamentos de todos os fatores de produção.
        \bigskip
        \item Dito de outra maneira, o lucro econômico deste tipo de firma é igual a zero.
        \bigskip
        \item Se o grau de homogeneidade fosse maior (menor) que um, então, os pagamentos totais excederiam (seriam menores que) o valor do produto.
    \end{itemize}
\end{frame}

\begin{frame}{Princípio das produtividades marginais decrescentes}
    \begin{itemize}
        \item \textcolor{blue}{Princípio dos rendimentos físicos (produtividades marginais) decrescentes.} Quanto mais se utiliza um fator de produção $i$, \emph{ceteris paribus}, a contribuição deste fator para o aumento da produção tende a ser cada vez menor, ou seja, o produto físico marginal do fator de produção $i$ é estritamente decrescente com relação à quantidade utilizada deste fator.
        \bigskip
        \item Formalmente:
        \begin{equation}
            \frac{\partial PM_i(\mathbf{x})}{\partial x_i} = \frac{\partial^2 f(\mathbf{x})}{\partial x_i^2} = f_{ii} < 0, \qquad \forall i = 1, \dots, n.
        \end{equation}
    \end{itemize}
\end{frame}

\begin{frame}{Princípio das produtividades marginais decrescentes}
    \begin{itemize}
        \item Considerando uma função de produção com apenas dois fatores - capital e trabalho - temos:
        \begin{eqnarray*}
            \frac{\partial PM_k}{\partial k} &=& \frac{\partial^2 f(k,l)}{\partial k^2} = f_{kk} < 0, \\
            \frac{\partial PM_l}{\partial l} &=& \frac{\partial^2 f(k,l)}{\partial l^2} = f_{ll} < 0.
        \end{eqnarray*}
         
        \item A hipótese de produtividade marginal decrescente foi originalmente proposta pelo economista do século XIX Thomas Malthus, que temia que o aumento rápido da população resultasse em uma menor produtividade do trabalho.
        \bigskip
        \item Suas predições pessimistas para o futuro da humanidade fizeram com que a economia ficasse conhecida como ``ciência sombria''.
    \end{itemize}
\end{frame}

\begin{frame}{Princípio das produtividades marginais decrescentes}
    \begin{figure}
        \centering
        \includegraphics[width=0.3\textwidth]{./figures/aula14_fig2.jpg}
        \caption{Thomas Robert Malthus (1766 - 1834). Fonte: \href{https://en.wikipedia.org/wiki/Thomas_Robert_Malthus}{Wikipedia}.}
        \label{fig3}
    \end{figure}
\end{frame}

\begin{frame}{Princípio das produtividades marginais decrescentes}
    \begin{itemize}
        \item Uma análise mais cuidadosa da função de produção sugere que tais predições pessimistas podem não ser corretas.
        \bigskip
        \item Variações na produtividade marginal do trabalho ao longo do tempo dependem não só de como o fator de produção trabalho está crescendo mas, também, de mudanças nos outros insumos (e.g., capital).
        \bigskip
        \item Ou seja, precisamos nos preocupar também com $\partial PM_l/\partial k = f_{lk}$.
        \bigskip
        \item Na maioria dos casos, $f_{lk} > 0$, portanto, a diminuição da produtividade do trabalho à medida que \emph{ambos $l$ e $k$ aumentam} pode ser uma conclusão precipitada.
        \bigskip
        \item De fato, a produtividade do trabalho parece ter aumentado significativamente desde a época de Malthus, principalmente porque os aumentos nos insumos de capital (combinado a melhorias tecnológicas) compensou o impacto do declínio da produtividade marginal.
    \end{itemize}
\end{frame}

\begin{frame}{Princípio das produtividades marginais decrescentes}
    \begin{itemize}
        \item Suponha que nossa função de produção $f(k,l)$, além de satisfazer o princípio das produtividades marginais decrescentes, apresente retornos constantes de escala.
        \bigskip
        \item Portanto, temos que (sem perda de generalidade):
        \begin{eqnarray*}
            f(tk, tl) &=& tf(k,l), \\
            \partial^2 f/\partial k^2 &<& 0.
        \end{eqnarray*}
    \end{itemize}
\end{frame}

\begin{frame}{Princípio das produtividades marginais decrescentes}
    \begin{itemize}
        \item Como a função produção é homogênea de grau um, pelo primeiro Teorema que vimos hoje, sua derivada é homogênea de grau zero, ou seja:
        \[
        \frac{\partial f(tk, tl)}{\partial k} = \frac{\partial f(k,l)}{\partial k}.
        \]
         
        \item Aplicando o teorema de Euler a $\partial f/\partial k$, temos:
        \[
        0 \frac{\partial f(k,l)}{\partial k} = k \frac{\partial^2 f(k,l)}{\partial k^2} + l \frac{\partial^2 f(k,l)}{\partial k \partial l}.
        \]
         
        \item Portanto:
        \[
        f_{kl} = \frac{\partial^2 f(k,l)}{\partial k \partial l} = -\frac{k}{l}\frac{\partial^2 f(k,l)}{\partial k^2} > 0
        \]
         
        \item Essa derivada parcial cruzada positiva significa que o produto marginal de um fator aumenta quando o outro fator aumenta - \textcolor{blue}{lei de Wicksell}.
    \end{itemize}
\end{frame}

\begin{frame}{Exercícios}
    \begin{itemize}
        \item Mostre que as seguintes funções são homogêneas e verifique o teorema de Euler:     \bigskip
        \begin{enumerate}
            \item $f(x_1, x_2) = x_1x_2^2$.     \bigskip
            \item $f(x_1,x_2) = x_1x_2 + x_2^2$.     \bigskip
            \item $f(x_1,x_2) = x_1$.
        \end{enumerate}
    \end{itemize}
\end{frame}

\begin{frame}{Propriedades de funções homogêneas}
    \begin{itemize}
        \item Homogeneidade é uma hipótese forte para uma função de produção e, especialmente, para uma função de utilidade.
        \bigskip
        \item Agora iremos considerar as consequências de adotarmos uma função homogênea ao responder as seguintes questões:     \bigskip
        \begin{enumerate}
            \item O que podemos afirmar acerca dos conjuntos de nível de uma função homogênea?\medskip
             
            \item Quais propriedades analíticas desejáveis uma função homogênea possui?     \bigskip
        \end{enumerate}
         
        \item A propriedade geométrica básica de uma função homogênea é uma consequência direta da definição de homogeneidade.
        \bigskip
        \item Seja $f: X \rightarrow \mathbb{R}$ uma função homogênea de grau $k$.
        \bigskip
        \item O conjunto de nível-$\alpha$ da função $f$ é definido por:
        \begin{equation}
            L(\alpha) \equiv \{x \in X| f(x) = \alpha\}.
            \label{eq4}
        \end{equation}
    \end{itemize}
\end{frame}

\begin{frame}{Propriedades de funções homogêneas}
    \begin{itemize}
        \item Seja $x_\alpha$ um ponto em $L(\alpha)$, e considere o ponto $t x_\alpha$ (com $t>0$) obtido ao nos movermos ao longo do raio que passa pela origem e por $x_\alpha$.
        \bigskip
        \item Então, $f(x_\alpha) = \alpha$ e, pela homogeneidade de $f$, temos que:
        \[
          f(t x_\alpha) = t^k f(x_\alpha) = t^k \alpha.  
        \]
         
        \item Portanto, podemos concluir que $t x_\alpha \in L(t^k \alpha)$ se $x_\alpha \in L(\alpha)$.
        \bigskip
        \item De forma análoga, se $y \in L(t^k \alpha)$, então, $(1/t)y$ está sobre o conjunto de nível $L(\alpha)$ pelo mesmo argumento.
        \bigskip
        \item Portanto, \textcolor{purple}{os conjuntos de níveis de funções homogêneas são expansões ou contrações radiais uns dos outros} - Figura \ref{fig4}.
    \end{itemize}
\end{frame}

\begin{frame}{Propriedades de funções homogêneas}
    \begin{figure}
        \centering
        \includegraphics[width=0.6\textwidth]{./figures/aula14_fig3.PNG}
        \caption{Conjuntos de nível de uma função homogênea. Fonte: De la Fuente (2000).}
        \label{fig4}
    \end{figure}
\end{frame}

\begin{frame}{Propriedades de funções homogêneas}
    \begin{itemize}
        \item Uma consequência da observação anterior pode ser enunciada no seguinte teorema:     \bigskip
        \begin{teo}
            Seja $q = f(\mathbf{x})$ uma função homogênea de classe $\mathcal{C}^1$ no ortante positivo. Os planos tangentes aos conjuntos de nível de $f$ possuirão inclinações constantes ao longo de cada raio a partir da origem.
        \end{teo}
    \end{itemize}
\end{frame}

\begin{frame}{Propriedades de funções homogêneas}
    \begin{figure}
        \centering
        \includegraphics[width=0.45\textwidth]{./figures/aula14_fig4.PNG}
        \caption{TTS de uma função homogênea é constante ao longo dos raios a partir da origem. Fonte: Simon e Blume (2004).}
        \label{fig5}
    \end{figure}
\end{frame}

\begin{frame}{Propriedades de funções homogêneas}
    \begin{itemize}
        \item O Teorema apenas enunciado tem consequências importantes para funções utilidade e de produção.
        \bigskip
        \item Por exemplo, suponha que $U(\mathbf{x})$ seja uma função utilidade homogênea.
        \bigskip
        \item A solução geométrica usual para o problema primal de maximização de utilidade do consumidor prediz que no ponto máximo, a curva de nível de $U$ é tangente à reta orçamentária - Figura \ref{fig6}.
        \bigskip
        \item Analiticamente, no ponto de máximo, a inclinação da curva de nível (ou TMS) é igual à inclinação da reta orçamentária.
    \end{itemize}
\end{frame}

\begin{frame}{Propriedades de funções homogêneas}
    \begin{figure}
        \centering
        \includegraphics[width=0.5\textwidth]{./figures/aula14_fig5.PNG}
        \caption{Problema primal de maximização de utilidade. Fonte: Simon e Blume (2004).}
        \label{fig6}
    \end{figure}
\end{frame}

\begin{frame}{Propriedades de funções homogêneas}
    \begin{itemize}
        \item Se aumentarmos a renda deste consumidor por um fator $r$ mantendo os preços unitários dos bens constantes, a reta orçamentária será deslocada paralelamente como na Figura \ref{fig6}.
        \bigskip
        \item A inclinação da reta orçamentária permanecerá a mesma.
        \bigskip
        \item A solução para o novo problema de otimização ocorre no ponto em que a TMS é igual à razão entre os preços dos bens.
        \bigskip
        \item Como a função utilidade é homogênea, este ponto será dado pela interseção entre a nova restrição orçamentária e o raio a partir da origem que passa pelo ponto ótimo original $x(I_0)$, pelo teorema que acabamos de ver.
        \bigskip
        \item Portanto, a curva parametrizada $I \rightarrow x(I)$ como na Figura \ref{fig6} indica que a cesta de consumo demandada para diferentes níveis de renda é chamada de \textcolor{purple}{trajetória (caminho) de expansão da renda}.
        
    \end{itemize}
\end{frame}

\begin{frame}{Propriedades de funções homogêneas}
    \begin{itemize}
        \item Acabamos de mostrar que a trajetória de expansão da renda de uma função utilidade homogênea é um \textbf{raio a partir da origem}.
        \bigskip
        \item Esta propriedade de funções homogêneas, implicada pelo teorema enunciado, é conhecida como \textcolor{purple}{homoteticidade}, que estudaremos posteriormente.
    \end{itemize}
\end{frame}

\subsection{Utilidade ordinal $\times$ utilidade cardinal}
\begin{frame}{Utilidade ordinal $\times$ utilidade cardinal}
    \begin{itemize}
        \item Funções homogêneas possuem propriedades que as tornam formas funcionais úteis para funções utilidade e de produção.
        \bigskip
        \item No entanto, o conceito moderno de utilidade é uma teoria \textcolor{purple}{ordinal}, e não \textcolor{blue}{cardinal}.
        \bigskip
        \item Homogeneidade, por sua vez, é uma propriedade cardinal, e não ordinal.
        \bigskip
        \item Posteriormente analisaremos uma classe mais ampla de funções que possuem as mesmas propriedades ordinais - \textcolor{purple}{funções homotéticas}.
    \end{itemize}    
\end{frame}

\begin{frame}{Utilidade ordinal $\times$ utilidade cardinal}
    \begin{itemize}
        \item Uma função utilidade provê uma mensuração do nível de satisfação associado a cada cesta de consumo.
        \bigskip
        \item No entanto, economistas não acreditam que um número real pode ser atribuído a cada cesta de consumo de maneira a expressar (em utils) o nível de satisfação de um consumidor com aquela cesta de consumo.
        \bigskip
        \item Economistas acreditam que consumidores possuem preferências bem comportadas sobre cestas de consumo e que, dadas duas cestas quaisquer, um consumidor pode estabelecer uma relação de preferência binária entre as duas.
        \bigskip
        \item Apesar de trabalharmos com funções utilidade, estamos interessados nos conjuntos de nível de tais funções, e não com o número real que a função utilidade associa a um conjunto de nível qualquer.
    \end{itemize}    
\end{frame}

\begin{frame}{Utilidade ordinal $\times$ utilidade cardinal}
    \begin{itemize}
        \item Na teoria da utilidade, estes conjuntos de nível são denominados \textcolor{purple}{conjuntos de indiferença}, ou curvas de indiferença quando os conjuntos de nível são curvas.
        \bigskip
        \item Uma propriedade de funções utilidades é chamada \textcolor{purple}{ordinal} se depende apenas do formato e posição dos conjuntos de indiferença de um consumidor.
        \bigskip
        \item Por outro lado, uma propriedade é denominada \textcolor{blue}{cardinal} se também depende da quantidade absoluta de utilidade que a função utilidade associa a cada um dos conjuntos de indiferença.
        \bigskip
        \item Neste contexto, duas funções são \textcolor{purple}{equivalentes} se possuem exatamente os mesmos conjuntos de indiferença, apesar de poderem associar diferentes números reais para um conjunto de indiferença qualquer.        
    \end{itemize}    
\end{frame}

\begin{frame}{Utilidade ordinal $\times$ utilidade cardinal}
    \begin{itemize}
        \item Exemplo, seja $u(x,y)$ uma função utilidade em $\mathbb{R}_+^2$. Defina $v(x,y)$ como a função utilidade dada por $u(x,y) + 1$.
        \bigskip
        \item Essas duas funções possuem exatamente as mesmas curvas de indiferença.
        \bigskip
        \item A função $v$ associa um número real que é maior em uma unidade que o número que a função $u$ associa a cada curva de indiferença.
        \bigskip
        \item Por exemplo, a curva de indiferença $\{u = 13\}$ coincide com a curva de indiferença $\{v = 14\}$.
        \bigskip
        \item As funções $u$ e $v$ representam as mesmas preferências e, portanto, são equivalentes.
    \end{itemize}    
\end{frame}

\begin{frame}{Utilidade ordinal $\times$ utilidade cardinal}
    \begin{itemize}
        \item A função utilidade $w(x,y) = [u(x,y)]^2$ também é equivalente a $u$.
        \bigskip
        \item Se $w(x_1, y_1) = w(x_2, y_2) = a$, então, $u(x_1, y_1) = u(x_2, y_2) = \sqrt{a}$.
        \bigskip
        \item As funções utilidade $u$ e $w$ possuem as mesmas curvas de indiferença, apenas atribuem diferentes números reais a elas.
        \bigskip
        \item Se $g_1(z) = z + 1$ e $g_2(z) = z^2$, então, podemos escrever $v = g_1 \circ u$ e $w = g_2 \circ u$.
        \bigskip
        \item Dizemos que $v$ e $w$ são \textcolor{purple}{transformações monotônicas} de $u$.
    \end{itemize}    
\end{frame}

\begin{frame}{Utilidade ordinal $\times$ utilidade cardinal}
    \begin{defi}[Transformação monotônica]
        Seja $I$ um intervalo sobre a reta real. Então, $g: I \to \mathbb{R}$ é uma \textcolor{purple}{transformação monotônica} de $I$ se $g$ é uma função estritamente crescente em $I$. Além disso, se $g$ é uma transformação monotônica e $u$ é uma função real de $n$ variáveis, então, dizemos que:
        \[
          g \circ u: x \mapsto g(u(x))  
        \]
        é uma \textcolor{purple}{transformação monotônica} de $u$.
    \end{defi}
    \begin{itemize}
        \item Obviamente, se $g$ é uma função diferenciável, então $g$ será uma transformação monotônica se $g'(x) > 0$ para qualquer $x \in I$.
    \end{itemize}
\end{frame}

\begin{frame}{Utilidade ordinal $\times$ utilidade cardinal}
    \begin{itemize}
        \item Uma característica de funções é dita \textcolor{purple}{ordinal} se qualquer transformação monotônica de uma função com esta característica, ainda possui esta característica.
        \bigskip
        \item Propriedades \textcolor{blue}{cardinais} não são preservadas por transformações monotônicas.
        \bigskip
        \item Observação: em análises de funções de produção, nos preocupamos com o número que a função de produção associa a uma isoquanta qualquer. O nível de produto para cada combinação de insumos possui uma interpretação significativa economicamente.
        \bigskip
        \item Dito de outra forma, a distinção entre cardinal e ordinal não é uma preocupação quando estamos tratando de funções de produção.
    \end{itemize}    
\end{frame}

\section{Funções homotéticas}
\begin{frame}{Funções homotéticas}
    \begin{itemize}
        \item Como acabamos de ver, homogeneidade é uma propriedade cardinal, e não ordinal.
        \bigskip
        \item Exemplos: as funções $g_1(z) = z^3 + z$ e $g_2(z) = z + 1$ são transformações monotônicas, mas a aplicação destas transformações à função homogênea $u(x,y) = xy$ leva a funções que não são homogêneas.
        \bigskip
        \item No entanto, muitas das propriedades que tornam funções homogêneas úteis em teoria da utilidade são propriedades ordinais:
        \bigskip
        \begin{enumerate}
            \item Conjuntos de nível são expansões ou contrações radiais uns dos outros.
            \bigskip
            \item A inclinação dos conjuntos de nível é constante ao longo de raios a partir da origem.
        \end{enumerate}
        \bigskip
        \item Estas são duas propriedades claramente ordinais: dizem respeito apenas à forma e inclinação das curvas de nível, sem preocupar-se com os números atribuídos a estes conjuntos de nível.
    \end{itemize}
\end{frame}

\begin{frame}{Funções homotéticas}
    \begin{itemize}
        \item Definiremos, então, uma classe de funções ordinais - uma classe que possui todas as propriedades ordinais que funções homogêneas possuem.
    \end{itemize}
    \begin{defi}[Funções homotéticas]
        Uma função $v: \mathbb{R}_+^n \to \mathbb{R}$ é dita \textcolor{purple}{homotética} se é uma transformação monotônica de uma função homogênea.

        Isto é, se existe uma transformação monotônica $z \mapsto g(z)$ do $\mathbb{R}_+$ e uma função homogênea $u: \mathbb{R}_+^n \to \mathbb{R}_+$ tal que:
        \[
          v(\mathbf{x}) = g(u(\mathbf{x})),  
        \]
        para todo $\mathbf{x}$ no domínio da função.
    \end{defi}
    \begin{itemize}
        \item Pela definição, percebe-se que homoteticidade é uma propriedade ordinal: uma transformação monotônica de uma função homotética é, também, uma função homotética.
    \end{itemize}
\end{frame}

\begin{frame}{Caracterizando funções homotéticas}
    \begin{itemize}
        \item Discutiremos, agora, duas das principais propriedades ordinais de funções homogêneas.
        \bigskip
        \item Veremos que estas propriedades caracterizam funções de utilidade homotéticas.
        \bigskip
        \item A primeira propriedade é que conjuntos de nível são expansões ou contrações radiais uns dos outros.
    \end{itemize}
\end{frame}

\begin{frame}{Funções homotéticas}
    \begin{defi}[Funções monotônicas]
        $u: \mathbb{R}_+^n \to \mathbb{R}$ é uma \textcolor{purple}{função monotônica} se $\forall x, y \in \mathbb{R}_+^n$:
        \[
          x \geq y \Rightarrow u(x) \geq u(y).  
        \]

        A função $u$ é \textcolor{purple}{estritamente monotônica} se $\forall x, y \in \mathbb{R}_+^n$:
        \[
          x > y \Rightarrow u(x) > u(y).  
        \]
    \end{defi}
\end{frame}

\begin{frame}{Funções homotéticas}
    \begin{itemize}
        \item Monotonicidade e monotonicidade estrita são propriedades naturais de funções utilidade.        
    \end{itemize}
    \begin{teo}
        Seja $u: \mathbb{R}_+^n \to \mathbb{R}$ uma função estritamente monotônica. Então, $u$ é homotética se, e somente se, para todo $x$ e $y$ em $\mathbb{R}_+^n$:
        \begin{equation}
            u(x) \geq u(y) \iff u(\alpha x) \geq u(\alpha y), \qquad \forall \alpha>0.
        \end{equation}
    \end{teo}
\end{frame}

\begin{frame}{Funções homotéticas}
    \begin{itemize}
        \item A segunda propriedade ordinal de homogeneidade é que a inclinação de conjuntos de nível é constante ao longo de raios a partir da origem.
        \bigskip         
        \item Esta propriedade fornece uma condição necessária baseada no cálculo para homoteticidade (da mesma forma que o teorema de Euler o faz para homogeneidade).
    \end{itemize}
    \begin{teo}
        Seja $u$ uma função $\mathcal{C}^1$ sobre $\mathbb{R}_+^n$. Se $u$ é homotética, então, as inclinações dos planos tangentes aos conjuntos de nível de $u$ são constantes ao longo dos raios a partir da origem.

        Em outras palavras, para quaisquer $i,j$ e para todo $x \in \mathbb{R}_+^n$:
        \begin{equation}
            \frac{\frac{\partial u}{\partial x_i}(t\mathbf{x})}{\frac{\partial u}{\partial x_j}(t\mathbf{x})} = \frac{\frac{\partial u}{\partial x_i}(\mathbf{x})}{\frac{\partial u}{\partial x_i}(\mathbf{x})}, \qquad \forall t>0.
            \label{cond15}
        \end{equation}
    \end{teo}            
\end{frame}

\begin{frame}{Funções homotéticas}
    \begin{itemize}
        \item O teorema anterior enuncia que se $u$ é uma função homotética, então, sua taxa marginal de substituição é uma função homogênea de grau zero.
        \bigskip
        \item De fato, a condição enunciada é, também, uma condição suficiente para mostrar que uma dada função é homotética.
        \bigskip
        \item Alguns textos definem uma função homotética caso sua taxa marginal de substituição seja homogênea de grau zero.        
    \end{itemize}    
    \begin{teo}
        Seja $u$ uma função $\mathcal{C}^1$ sobre $\mathbb{R}_+^n$. Se a condição \ref{cond15} é válida para qualquer $x \in \mathbb{R}_+^n$, para todo $t > 0$ e quaisquer $i,j$, então, $u$ é homotética.
    \end{teo}
\end{frame}

\section{Bibliografia}
\begin{frame}{\emoji{books} Bibliografia}
    \begin{itemize}                
        \item CHIANG, A.C.; WAINWRIGHT, K. Matemática para economistas. Rio de Janeiro: Elsevier, 2006\medskip
        \item DE LA FUENTE, Á. Mathematical methods and models for economists. Cambridge University Press, Cambridge, UK, 2000\medskip
        \item NICHOLSON, W.; SNYDER C. Teoria microeconômica: Princípios básicos e aplicações. Cengage Learning Brasil, 2019. Disponível em: \href{https://app.minhabiblioteca.com.br/books/9788522127030/}{app.minhabiblioteca.com.br/books/9788522127030/}\medskip
        \item SIMON, C.P.; BLUME, L. Matemática para economistas. Porto Alegre: Bookman, 2004\medskip
        \item SYDSÆTER, K.; HAMMOND, P.J.; STRØM, A.; CARVAJAL, A. Essential mathematics for economic analysis. 5th.ed. Harlow, UK: Pearson Education Limited, 2016
    \end{itemize}
\end{frame}
\end{document}