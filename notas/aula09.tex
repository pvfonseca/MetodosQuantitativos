\documentclass[12pt]{article}
\usepackage{bm}% bold math
\usepackage[english,brazilian]{babel}
\usepackage[utf8]{inputenc}
\usepackage{graphicx}
\usepackage{amsmath}
\usepackage{amssymb}
\usepackage{setspace}
\usepackage{caption}
\usepackage{xcolor}
\usepackage{setspace}
\newtheorem{teo}{Teorema}
\newtheorem{defi}{Definição}
\usepackage{hyperref}
\usepackage{alertmessage}
\usepackage[margin=1.0in]{geometry}
\hypersetup{colorlinks=true, linkcolor=blue, citecolor=blue, filecolor=blue, urlcolor=blue,
            pdfauthor={Nome do Autor},
            pdftitle={Título do Projeto},
            pdfsubject={Assunto do Projeto},
            pdfkeywords={palavra-chave, palavra-chave, palavra-chave},
            pdfproducer={Latex},
            pdfcreator={pdflatex}}

\title{Estática comparativa e teorema do envelope}
\author{}
\date{}

\begin{document}

\begin{center}
    \textbf{UNIVERSIDADE DO ESTADO DE SANTA CATARINA \\ Centro de Ciências da Administração e Socioeconômicas \\ Departamento de Ciências Econômicas}
\end{center}
    
\vskip 1em
    
\noindent\textbf{Disciplina:} Métodos Quantitativos em Economia I
    
\noindent\textbf{Docente:} \href{https://pvfonseca.github.io}{Paulo Victor da Fonseca} 
    
\noindent\textbf{Contato:} \href{mailto:paulo.fonseca@udesc.br}{paulo.fonseca@udesc.br}
    
\noindent\textbf{Página da disciplina:} \href{https://pvfonseca.github.io/teaching/metodosquant}{Métodos Quantitativos I}
    

\alertwarning{O texto que segue não tem a menor pretensão de originalidade. Ele serve apenas como registro dos principais princípios, conceitos e técnicas analíticas que são trabalhados em sala de aula.}

\onehalfspacing
\section{Estática comparativa}

Modelos econômicos possuem dois tipos de variávies: \textcolor{purple}{variávies endógenas}, cujos valores são explicados pelo próprio modelo, e \textcolor{purple}{variáveis exógenas}, cujos valores são tomados como dados (de fora do modelo).

Os valores de solução que obtemos para as variáveis endógenas irão, tipicamente, depender dos valores das variáveis exógenas e uma parte central da análise econômica é, frequentemente, mostrar como os valores da solução das variáveis endógenas variam quando as variáveis exógenas são alteradas - \textcolor{blue}{estática comparativa}.

\noindent\textbf{Exemplos:} Modelo Keynesiano simples de determinação da renda agregada; modelo de mercado linear; impostos sobre produto em um monopólio.

\noindent\underline{Na lousa}.

A análise de estática comparativa que acabamos de ver é típica em economia. Usamos as relações básicas do modelo para derivar uma \textcolor{purple}{equação fundamental} contendo apenas a variável endógena que buscamos a solução, a variável exógena e os parâmetros do modelo. O próximo passo é encontrar uma equação para a variável endógena em função da variável exógena.

A questão que buscamos responder em um exercício de estática comparativa é: qual o efeito que uma variação em uma variável exógena tem sobre a variável endógena do modelo? Estamos interessados nos sinais algébricos destas expressões pois eles nos dão uma informação \textcolor{purple}{qualitativa} sobre a direção da variação no valor de equilíbrio seguindo uma mudança na variável exógena. Esse efeito pode ser analisado graficamente, no entanto, a abordagem algébrica nos permite determinar a força e direção dos efeitos de estática comparativa dados os parâmetros do modelo.

\subsection{Análise de estática comparativa generalizada}

\subsubsection{Uma variável endógena e uma variável exógena}

Vamos supor que temos um modelo econômico para o qual a solução de equilíbrio é dada por uma equação da forma:
\begin{equation}
    f(x^*, \alpha) = 0,
    \label{eq1}
\end{equation}
onde $x^*$ é o valor de equilíbrio da variável endógena. E assume-se que $f$ é uma função diferenciável.

O procedimento para análise de estática comparativa é como segue: assuma que é possível resolver a equação (\ref{eq1}) para $x^*$ como uma função diferenciável de $\alpha$ (veremos brevemente as condições sob as quais é possível fazer isso), e escreva a solução como $x^*(\alpha)$. Portanto, temos:
\begin{equation}
    f(x^*(\alpha), \alpha) = 0.
    \label{eq2}
\end{equation}

Então, diferencie esta função com relação a $\alpha$ para obter:
\begin{equation}
    f_x \frac{dx^*}{d\alpha} + f_\alpha = 0.
\end{equation}

Então, resolvendo esta equação sob a hipótese de que $f_x \neq 0$, obtemos:
\begin{equation}
    \frac{dx^*}{d\alpha} = -\frac{f_\alpha(x^*,\alpha)}{f_x(x^*,\alpha)}.
\end{equation}

\noindent\textbf{Exercício:} Considere a função implícita $f(x^*(\alpha),\alpha) = \ln x^* - 2\alpha^2 = 0$. Encontre o valor de $dx^*/d\alpha$.

\vskip 1em
\noindent\textbf{Aplicação econômica: efeito de uma variação da renda no mercado de um bem}

Suponha que a função de demanda de mercado  para um bem é $D(p,y)$, onde $p$ é o preço e $y$ a renda agregada do consumidor. A função de oferta de mercado é $S(p)$. Então, o valor de equilíbrio do preço $p^*$ é dada pela condição:
\[
D(p^*,y) - S(p^*) = 0,
\]
onde $D - S$ é o excesso de demanda.

Portanto, temos a solução:
\[
\frac{dp^*}{dy} = -\frac{D_y}{D_p-S_p}.
\]
Suponha que $D_p<0$ (o bem não é de Giffen) e $S_p > 0$. Então, o efeito de um aumento da renda sobre o preço de equilíbrio depende se o bem é normal ($D_y>0$) ou inferior ($D_y \leq 0$). Se $D_p>0$ (bem de Giffen), o bem é, necessariamente, inferior ($D_y\leq 0$) e, portanto, o efeito de um aumento da renda sobre o preço vai depender do sinal de $D_p-S_p$.

\subsubsection{Estática comparativa com várias variáveis endógenas e exógenas}

Considere, inicialmente, o caso de duas variáveis de cada tipo. Como veremos, isso requer um equilíbrio ou condição de primeira ordem - uma função $f$ - para cada variável endógena. Portanto, a solução será dada pelas condições:
\begin{eqnarray*}
f^1(x_1^*, x_2^*, \alpha_1, \alpha_2) &=& 0, \\
f^2(x_1^*, x_2^*, \alpha_1, \alpha_2) &=& 0.
\end{eqnarray*}
Agora, as soluções de equilíbrio das variáveis endógenas dependem de ambas variáveis exógenas. Queremos derivar e determinar os sinais, se possível, das quatro derivadas existentes.

Assuma que seja possível resolver essas equações para $x_1^*$ e $x_2^*$ como funções diferenciáveis de $\alpha_1$ e $\alpha_2$. Então:
\begin{eqnarray*}
    f^1(x_1^*(\alpha_1, \alpha_2), x_2^*(\alpha_1, \alpha_2), \alpha_1, \alpha_2) &=& 0, \\
f^2(x_1^*(\alpha_1, \alpha_2), x_2^*(\alpha_1, \alpha_2), \alpha_1, \alpha_2) &=& 0.
\end{eqnarray*}

Diferenciando com relação a $\alpha_1$, obtemos:
\begin{eqnarray*}
    f_1^1 \frac{\partial x_1^*}{\partial \alpha_1} + f_2^1 \frac{\partial x_2^*}{\partial \alpha_1} + f^1_{\alpha_1} &=& 0, \\
    f_1^2 \frac{\partial x_1^*}{\partial \alpha_1} + f_2^2 \frac{\partial x_2^*}{\partial \alpha_1} + f^2_{\alpha_1} &=& 0.
\end{eqnarray*}

Em notação matricial:
\begin{equation*}
    \begin{bmatrix}
        f_1^1 & f_2^1 \\ f_1^2 & f_2^2
    \end{bmatrix} \begin{bmatrix}
        \partial x_1^*/\partial \alpha_1 \\ 
        \partial x_2^*/\partial \alpha_1
    \end{bmatrix} = \begin{bmatrix}
        -f_{\alpha_1}^1 \\ -f_{\alpha_1}^2
    \end{bmatrix}
\end{equation*}

Para resolver esse sistema linear, é necessário que a seguinte condição para o  determinante seja satisfeita:
\[
|D| = f_1^1 f_2^2 - f_2^1 f_1^2 \neq 0.
\]

Assumindo esta condição e usando a regra de Cramer, temos:
\begin{eqnarray*}
    \frac{\partial x_1^*}{\partial \alpha_1} &=& \frac{\begin{vmatrix}
        -f_{\alpha_1}^1 & f_2^1 \\ -f_{\alpha_1}^2 & f_2^2
    \end{vmatrix}}{|D|} = \frac{-(f_{\alpha_1}^1f_2^2 - f_{\alpha_1}^2f_2^1)}{|D|}, \\
    \frac{\partial x_2^*}{\partial \alpha_1} &=& \frac{\begin{vmatrix}
        f_1^1 & -f_{\alpha_1}^1  \\ f_1^2 & -f_{\alpha_1}^2 
    \end{vmatrix}}{|D|} = \frac{-(f_{\alpha_1}^2f_1^1 - f_{\alpha_1}^1f_1^2)}{|D|}.
\end{eqnarray*}
Para determinar os sinais algébricos destas derivadas, precisamos saber o sinal de todas as derivadas parciais envolvidas. Além disso, os numeradores e denominadores envolvem diferenças entre dois termos e, portanto, precisamos saber ou assumir algo a respeito das magnitudes destes termos para definir os sinais das soluções.

\noindent\textbf{Aplicação econômica:} Modelo IS-LM (Na lousa).

\subsubsection{Método geral de estática comparativa}

\begin{defi}{Método generalizado de estática comparativa}

Dado que o sistema de $n$ equações:
\begin{eqnarray*}
    f^1(x_1^*, x_2^*, \dots, x_n^*; \alpha_1, \dots, \alpha_m) &=& 0 \\
    f^2(x_1^*, x_2^*, \dots, x_n^*; \alpha_1, \dots, \alpha_m) &=& 0 \\
    \vdots &\ddots & \vdots \\
    f^n(x_1^*, x_2^*, \dots, x_n^*; \alpha_1, \dots, \alpha_m) &=& 0
\end{eqnarray*}
possa ser resolvido para os valores de equilíbrio das $n$ variáveis endógenas $x_1^*, \dots, x_n^*$ como funções diferenciáveis das $m$ variáveis exógenas $\alpha_1, \dots, \alpha_m$, temos que:
\[
\frac{\partial x_i^*}{\partial \alpha_j} = \frac{|F_{ij}|}{|F|}, \qquad i=1,\dots,n; \quad j=1,\dots,m,
\]
onde $|F|\neq 0$ é o determinante:
\[
|F| = \begin{vmatrix}
    f_1^1 & f_2^1 & \dots & f_n^1 \\
    f_1^2 & f_2^2 & \dots & f_n^2 \\
    \vdots & \vdots & \ddots & \vdots \\
    f_1^n & f_2^n & \dots & f_n^n
\end{vmatrix},
\]
e $|F_{ij}|$ é obtido ao substituírmos a $i$-ésima coluna de $|F|$ pela $j$-ésima coluna da matriz $n\times m$:
\[
\begin{bmatrix}
    -f_{\alpha_1}^1 & -f_{\alpha_2}^1 & \dots & -f_{\alpha_m}^1 \\
    -f_{\alpha_1}^2 & -f_{\alpha_2}^2 & \dots & -f_{\alpha_m}^2 \\
    \vdots & \vdots & \ddots & \vdots \\
    -f_{\alpha_1}^n & -f_{\alpha_2}^n & \dots & -f_{\alpha_m}^n
\end{bmatrix}.
\]
\label{def1}
\end{defi}

Note que:
\begin{itemize}
    \item Devemos assumir que $|F| \neq 0$.
    \item As derivadas parciais são avaliadas no ponto de equilíbrio inicial e, portanto, são valores reais dados.
    \item Quando o sistema de equações representa as CPOs de um problema de maximização ou minimização, as CSOs para este problema determinarão o sinal de $|F|$, dado que são o determinante da Hessiana orlada na análise de condições de segunda ordem (veremos mais adiante na disciplina).
\end{itemize}

Até agora assumimos que o sistema de equações da Definição \ref{def1} resultam em soluções para as $n$ variáveis endógenas como funções diferenciáveis das $m$ variáveis exógenas. O teorema que determina as condições sob as quais isso pode ser assumido é, então, enunciado.

\begin{teo}[Teorema da Função Implícita]

    Dado um sistema de equações:
    \begin{eqnarray*}
        f^1(x_1^*, x_2^*, \dots, x_n^*; \alpha_1, \dots, \alpha_m) &=& 0 \\
    f^2(x_1^*, x_2^*, \dots, x_n^*; \alpha_1, \dots, \alpha_m) &=& 0 \\
    \vdots &\ddots & \vdots \\
    f^n(x_1^*, x_2^*, \dots, x_n^*; \alpha_1, \dots, \alpha_m) &=& 0,
    \end{eqnarray*}
    seja $(x_1^*, \dots, x_n^*; \alpha_1^0, \dots, \alpha_m^0)$ um ponto que satisfaça essas equações, e que as funções possuam derivadas parciais contínuas até uma ordem $r$, sobre algum conjunto aberto de pontos no $\mathbb{R}^{n+m}$ ao redor de $(x_1^*, \dots, x_n^*; \alpha_1^0, \dots, \alpha_m^0)$. Se o determinante do Jacobiano:
    \[
|F| = \begin{vmatrix}
    f_1^1 & f_2^1 & \dots & f_n^1 \\
    f_1^2 & f_2^2 & \dots & f_n^2 \\
    \vdots & \vdots & \ddots & \vdots \\
    f_1^n & f_2^n & \dots & f_n^n
\end{vmatrix} \neq 0,
    \]
    onde $f_i^k \equiv \partial f^k/\partial x_i$, e essas derivadas são avaliadas em $(x_1^*, \dots, x_n^*; \alpha_1^0, \dots, \alpha_m^0)$, então, o sistema de equações define $x_1, \dots, x_n$ como funções dos $\alpha_j$'s em alguma vizinhança do ponto  $(x_1^*, \dots, x_n^*; \alpha_1^0, \dots, \alpha_m^0)$. Nesta vizinhança, temos:
    \[
\begin{bmatrix}
    f_1^1 & f_2^1 & \dots & f_n^1 \\
    f_1^2 & f_2^2 & \dots & f_n^2 \\
    \vdots & \vdots & \ddots & \vdots \\
    f_1^n & f_2^n & \dots & f_n^n
\end{bmatrix}\begin{bmatrix}
    \partial{x_1^*}/\partial \alpha_j \\
    \partial{x_2^*}/\partial \alpha_j \\
    \vdots \\
    \partial{x_n^*}/\partial \alpha_j \\
\end{bmatrix} = \begin{bmatrix}
    -f_{\alpha_j}^1 \\
    -f_{\alpha_j}^2 \\
    \vdots \\
    -f_{\alpha_j}^n \\
\end{bmatrix},
    \]
    para cada $j = 1, \dots, m$.
\end{teo}

\section{Teorema do envelope}

Em análises de estática comparativa em contextos de problemas de otimização com restrição, é comum usarmos uma abordagem baseada no \textcolor{purple}{teorema do envelope} - em adição ou substituição ao teorema da função implícita.

Considere o seguinte problema de otimização com restrição:
\begin{eqnarray}
    & \max_{\substack{x_1, x_2}} & f(x_1, x_2; \alpha) \nonumber \\
    & \text{s.r.} & g(x_1, x_2; \alpha) = 0, \nonumber
\end{eqnarray}
onde $\alpha$ é uma variável exógena. A função Lagrangeana para este problema é:
\[
\mathcal{L}(x_1, x_2, \lambda; \alpha) = f(x_1, x_2; \alpha) + \lambda g(x_1, x_2; \alpha),    
\]
e as CPOs associadas:
\begin{eqnarray*}
    f_1(x_1^*, x_2^*; \alpha) + \lambda^* g_1(x_1^*, x_2^*; \alpha) &=& 0, \\
    f_2(x_1^*, x_2^*; \alpha) + \lambda^* g_2(x_1^*, x_2^*; \alpha) &=& 0, \\
    g(x_1^*, x_2^*; \alpha) &=& 0.
\end{eqnarray*}

Se as funções $f$ e $g$ possuem primeira e segunda derivadas contínuas e, além disso:
\[
|D| = \begin{vmatrix}
    0 & g_1 & g_2 \\
    g_1 & f_{11} + \lambda^*g_{11} & f_{12} + \lambda^*g_{12} \\
    g_2 & f_{21} + \lambda^*g_{21} & f_{22} + \lambda^*g_{22}
\end{vmatrix}  \neq 0,
\]
então, podemos aplicar o teorema da função implícita. Isto quer dizer que podemos obter as soluções para as variáveis endógenas como funções da variável exógena em uma vizinhança do ponto ótimo, de forma que o valor da função $f$ nessa mesma vizinhança é:
\[
f(x_1(\alpha), x_2(\alpha); \alpha) \equiv V(\alpha),
\]
e $V$ é uma \textcolor{purple}{função valor} para o problema de maximização.

Da mesma forma, podemos escrever a função Lagrangeana como função do parâmetro:
\[
\mathcal{L} = f(x_1(\alpha), x_2(\alpha); \alpha) + \lambda(\alpha) g(x_1(\alpha),x_2(\alpha);\alpha).
\]

Temos, então, que:
\[
\frac{d\mathcal{L}}{d\alpha} = (f_1 + \lambda g_1)\frac{d x_1}{d\alpha} + (f_2 + \lambda g_2)\frac{d x_2}{d\alpha} + g\frac{d\lambda }{d\alpha} + (f_\alpha + \lambda g_\alpha),
\]
que quando avaliada no ponto ótimo nos dá o seguinte resultado:
\[
\frac{d\mathcal{L}}{d\alpha} = f_\alpha + \lambda^* g_\alpha = \frac{\partial\mathcal{L}}{\partial\alpha},
\]
ou seja, mesmo que mudanças no valor de $\alpha$ induzam variações nos valores das variáveis endógenas, para pequenas mudanças no \textbf{ponto ótimo}, os efeitos dessas alterações na função Lagrangeana podem ser ignoradas pois as derivadas parciais da função Lagrangeana com relação às variáveis endógenas são nulas neste ponto.

O \textcolor{purple}{teorema do envelope} estabelece uma conexão entre as derivadas da função valor e as derivadas da função Lagrangeana, com relação a $\alpha$, no ponto ótimo. Portanto, para a função valor, temos:
\[
\frac{d V}{d\alpha} = f_1\frac{d x_1}{d\alpha} + f_2\frac{d x_2}{d\alpha} + f_\alpha,
\]
pelas CPOs, temos:
\[
\frac{d V}{d\alpha} = -\lambda^*\left(g_1 \frac{d x_1}{d\alpha} + g_2 \frac{d x_2}{d\alpha} \right) + f_\alpha
\]
no ponto ótimo.

A restrição do problema pode ser escrita como:
\[
g(x_1(\alpha), x_2(\alpha); \alpha) = 0,
\]
onde, como $x_i$ são as soluções ótimas e, portanto, satisfazem a restrição, esta expressão é satisfeita com igualdade. Portanto:
\[
g_1\frac{d x_1}{d\alpha} + g_2\frac{d x_2}{d\alpha} = -g_\alpha.
\]

Portanto, para este exemplo, o \textcolor{purple}{teorema do envelope} nos diz que:
\begin{equation}
    \frac{d V}{d\alpha} = f_\alpha + \lambda^* g_\alpha = \frac{\partial \mathcal{L}}{\partial\alpha}.
\end{equation}

\textbf{O teorema do envelope diz que podemos encontrar o efeito de uma variação na variável exógena sobre o valor ótimo da função objetivo simplesmente tomando a derivada parcial da função Lagrangeana com relação à variável exógena avaliada na solução ótima do problema}.

\begin{teo}[Teorema do Envelope]
    Dado o problema:
    \begin{eqnarray}
        \max && f(x_1, \dots, x_n; \alpha_1,\dots, \alpha_m), \\
        \text{s.r.} && g^1(x_1, \dots, x_n; \alpha_1, \dots, \alpha_m) \nonumber \\
        && \ddots \nonumber \\
        && g^K(x_1, \dots, x_n; \alpha_1, \dots, \alpha_m) \nonumber
    \end{eqnarray}
    e a função valor correspondente $V(\alpha_1, \dots, \alpha_m)$, e a função Lagrangeana $\mathcal{L} = f + \sum\lambda_kg^k$, temos que:
    \begin{equation}
        \frac{\partial V}{\partial\alpha_j} = \frac{\partial\mathcal{L}}{\partial\alpha_j} = f_{\alpha_j} + \sum_{k=1}^K \lambda_j g^k_{\alpha_j}, \qquad j=1,\dots,m.
    \end{equation}
\end{teo}

%\newpage
\begin{thebibliography}{}
\bibitem{chiang}
CHIANG, A.C.; WAINWRIGHT, K. Matemática para economistas. Rio de Janeiro: Elsevier, 2006.

\bibitem{hoy}
HOY, M.; LIVERNOIS, J.; McKENNA, C.; REES, R.; STENGOS, T. Mathematics for Economics. 3rd.ed. Cambridge, Massachusetts: The MIT Press, 2011.

\bibitem{silberberg}
SILBERBERG, E.; SUEN, W. The structure of economics: a mathematical analysis. 3rd.ed. Singapore: McGraw-Hill Higher Education, 2001.

\bibitem{sydsaeter}
SYDSÆTER, K.; HAMMOND, P.J.; STRØM, A.; CARVAJAL, A. Essential mathematics for economic analysis. 5th.ed. Harlow, UK: Pearson Education Limited, 2016.
\end{thebibliography}

\end{document}