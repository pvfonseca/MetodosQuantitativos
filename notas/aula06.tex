\documentclass[preprintnumbers,nofootinbib,amsmath,amssymb,12pt]{article}
\usepackage{dcolumn}% Align table columns on decimal point
\usepackage{bm}% bold math
\usepackage{amsthm}
\usepackage[english,brazilian]{babel}
\usepackage[utf8]{inputenc}
\usepackage{makeidx}
\usepackage{framed}
\usepackage{cmap}
\usepackage{graphicx}
\usepackage{float}
\usepackage[caption = false]{subfig}
\usepackage[T1]{fontenc}
\usepackage{amsmath,amssymb,amsfonts,textcomp}
\usepackage{setspace}
\usepackage{caption}
\usepackage{multicol}
\usepackage{multirow}
\usepackage{array}
%\usepackage{booktabs}
\usepackage{setspace}
\usepackage{lscape}
\usepackage{anysize}
\usepackage{hyperref}
\usepackage{booktabs,caption}
\usepackage{longtable}
\usepackage{chngcntr}
\usepackage[portuguese,linesnumbered,ruled]{algorithm2e}
\hypersetup{colorlinks=true, linkcolor=blue, citecolor=blue, filecolor=blue, pagecolor=blue, urlcolor=blue,
            pdfauthor={Nome do Autor},
            pdftitle={Título do Projeto},
            pdfsubject={Assunto do Projeto},
            pdfkeywords={palavra-chave, palavra-chave, palavra-chave},
            pdfproducer={Latex},
            pdfcreator={pdflatex}}
\marginsize{2cm}{2cm}{2cm}{2cm}
\renewenvironment{quote}{\list{}{\small \singlespacing \leftmargin=4cm\rightmargin=0cm}\item[]}{\endlist} 
\usepackage{natbib}
\newtheorem{alg}{Algoritmo}
\newtheorem{prop}{Proposição}
\newtheorem{teo}{Teorema}
\newtheorem{defi}{Definição}
\newtheorem{lema}{Lema}
\newtheorem{coro}{Corolário}
\newtheorem{hip}{Hipótese}
\newcommand{\algorithmfootnote}[2][\footnotesize]{%
  \let\old@algocf@finish\@algocf@finish% Store algorithm finish macro
  \def\@algocf@finish{\old@algocf@finish% Update finish macro to insert "footnote"
    \leavevmode\rlap{\begin{minipage}{\linewidth}
    #1#2
    \end{minipage}}%
  }%
}

\title{}
\author{}
\date{}

\begin{document}

\begin{center}
\textbf{UNIVERSIDADE DO ESTADO DE SANTA CATARINA \\ Centro de Ciências da Administração e Socioeconômicas \\ Departamento de Ciências Econômicas}
\end{center}

\vskip 1em

\noindent\textbf{Disciplina:} Métodos Quantitativos em Economia I

\noindent\textbf{Docente:} \href{https://pvfonseca.github.io}{Paulo Victor da Fonseca} 

\noindent\textbf{Contato:} \href{mailto:paulo.fonseca@udesc.br}{paulo.fonseca@udesc.br}

\noindent\textbf{Página da disciplina:} \href{https://pvfonseca.github.io/teaching/metodosquant}{Métodos Quantitativos I}

\hrulefill

\vskip 1em
\section{Otimização estática irrestrita: Aplicações econômicas}

\subsection{Maximização de lucros de uma firma competitiva uniproduto}

Suponha que uma firma produza em um mercado perfeitamente competitivo combinando os insumos capital ($K$) e trabalho ($L$), de acordo com a seguinte função de produção:
\begin{equation}
    Q = F(K,L).
    \label{eq1}
\end{equation}

Sejam $p$ o preço unitário do produto produzido, $r$ o custo unitário (taxa de aluguel) do capital e $w$ o custo unitário do trabalho (taxa salarial), assume-se que a firma seja tomadora de preços no mercado de bens e no mercado de insumos e, ainda, que $p, w$ e $r$ são constantes positivas.

O lucro, $\pi$, de produzir e vender $F(K,L)$ unidades é, então, dada pela diferença entre receita total e custo total, ou seja, dado pela seguinte função:
\begin{equation}
    \pi(K,L) = pF(K,L) - wL - rK.
    \label{eq2}
\end{equation}

Supondo que o objetivo da firma seja maximizar seus lucros, temos, então, o seguinte problema de otimização estática e sem restrições:
\begin{eqnarray}
&\max_{\{K,L\}}& pF(K,L) - wL - rK.
\end{eqnarray}

Se $F$ é uma função diferenciável e $\pi$ tem um ponto de máximo interior, ou seja, com $K>0$ e $L>0$, as condições necessárias de primeira ordem são dadas por:
\begin{eqnarray}
\pi_K(K,L) &\therefore& pF_K(K,L) - r = 0 \label{eq4} \\
\pi_L(K,L) &\therefore& pF_L(K,L) - w = 0 \label{eq5}
\end{eqnarray}

A equação (\ref{eq4}) nos diz que, no ponto de ótimo, o custo do capital $r$ deve equalizar o valor, ao preço unitário $p$, do produto marginal do capital. A equação (\ref{eq5}) possui uma interpretação similar para o trabalho.

Podemos, ainda, reescrever as equações (\ref{eq4})-(\ref{eq5}) da seguinte forma:
\begin{eqnarray}
F_K(K,L) &=& \frac{r}{p}, \label{eq6} \\
F_L(K,L) &=& \frac{w}{p}. \label{eq7}
\end{eqnarray}

A equação (\ref{eq6}) nos diz que, para obter lucro máximo, a firma deve escolher $K$ e $L$ que equalize a produtividade marginal do capital ao seu preço ``relativo'', $r/p$. De maneira análoga, temos a expressão para o trabalho.

Cabe ressaltar que as condições (\ref{eq4})-(\ref{eq5}) são apenas necessárias mas não suficientes para assegurar um ponto de máximo. Seja $(K^*, L^*)$ o par ordenado que satisfaz de forma simultânea o sistema citado, as condições suficientes de segunda ordem para um ponto de máximo são, então, dadas por:
\begin{eqnarray*}
F_{KK}(K^*, L^*) < 0, \qquad F_{LL}(K^*, L^*) < 0 \\
F_{KK}(K^*, L^*)F_{LL}(K^*, L^*) > [F_{KL}(K^*, L^*)]^2.
\end{eqnarray*}

\noindent\textbf{Exercícios.}
\begin{enumerate}
    \item Encontre os valores de capital e trabalho que maximizem a função lucro a seguir e verifique que o ponto é, de fato, um ponto de máximo local:
    \[
    \pi(K,L) = 12K^{1/2}L^{1/4} - 1,2K - 0,6L.
    \]
    
    \item Suponha uma firma que produza dois tipos de bens operando em um mercado de concorrência perfeita.
    
    Se os preços unitários das mercadorias produzidas são denotados por $p_1$ e $p_2$, a função receita total é dada por:
    \[
    R = p_1q_1 + p_2q_2.
    \]
    
    Suponha que a firma depara-se com uma função custo total quadrática\footnote{Note que os custos marginais de cada bem dependem da quantidade produzida da outra mercadoria. Dizemos, então, que as mercadorias estão tecnicamente relacionadas em termos de produção.}:
    \[
    C = 2q_1^2 + q_1q_2 + 2q_2^2.
    \]
    
    Encontre os valores produzidos de cada bem que maximizem a função lucro e mostre que o ponto é, de fato, um ponto de máximo relativo.
\end{enumerate}

\newpage
\subsection{Monopólio com discriminação de preços}
Um monopolista é capaz de dividir o mercado existente pelo bem que produz em dois submercados. Suponha que as funções de demanda inversas em cada submercado são dadas por:
\begin{eqnarray}
p_1 &=& 100-q_1, \nonumber \\
p_2 &=& 120 - 2q_2. \nonumber
\end{eqnarray}

Supõe-se que este monopolista produz seu bem usando uma única planta industrial, de forma que a função custo é dada pela seguinte expressão:
\begin{equation}
    C = 20(q_1 + q_2).
    \nonumber
\end{equation}

Portanto, o custo unitário de produção é constante em \$20, não há custo fixo e não distingue-se o produto dos dois bens no processo produtivo.

Temos, então, que a função lucro é dada por:
\begin{eqnarray*}
\pi(q_1, q_2) &=& p_1q_1 + p_2q_2 - C \\
&=& 80q_1 - q_1^2 + 100q_2 - 2q_2^2.
\end{eqnarray*}

Se o objetivo do monopolista discriminador de preços é maximizar sua função lucro temos, então, as condições necessárias de primeira ordem são dadas pelas seguintes expressões:
\begin{eqnarray*}
\pi_1 &\therefore& 80 - 2q_1 = 0, \\
\pi_2&\therefore& 100-4q_2 = 0.
\end{eqnarray*}
Portanto, as quantidades ótimas produzidas serão dadas por $(q_1^*, q_2^*) = (40, 25)$ o que implica que os preços nos dois submercados serão iguais a $(p_1, p_2) = (\$60, \$70)$ e o lucro máximo será de \$2.850.

Para verificar que o ponto crítico é, de fato, um ponto de ótimo obtemos a matriz Hessiana associada ao problema de otimização:
\[
|H| = \begin{vmatrix}
-2 & 0 \\0 & -4
\end{vmatrix},
\]
os determinantes dos menores principais líderes implicam que $|H_1| = -2<0$ e $|H_2| = 8> 0$ e, então, a matriz Hessiana é negativa definida o que assegura um ponto de máximo.

\newpage
\subsubsection{Problema de alocação do produto total}

Consideraremos, agora, o problema de alocação do produto total para esse monopolista discriminador de preços de uma maneira mais geral.

A função lucro, neste caso, é dada por:
\begin{equation}
    \pi(q_1, q_2) = R^1(q_1) + R^2(q_2) - c(q_1 + q_2),
    \label{eq10}
\end{equation}
onde $R^i$ é a função receita total no submercado $i \in\{1,2\}$, $c > 0$ é o custo unitário constante.

Supondo que o objetivo do monopolista discriminador é maximizar sua função lucro e as funções de receita total são diferenciáveis, então, as condições necessárias de primeira ordem são:
\begin{eqnarray}
\pi_1 &\therefore& RMg^1(q_1^*) - c = 0, \label{eq11} \\
\pi_2 &\therefore& RMg^2(q_2^*) - c = 0, \label{eq12}
\end{eqnarray}
onde $RMg^i$ é a receita marginal no submercado $i$.

Portanto, as receitas marginais nos dois submercados devem ser equalizadas:
\begin{equation}
    RMg^1(q_1^*) = RMg^2(q_2^*). \label{eq13}
\end{equation}

A intuição por trás da condição (\ref{eq13}) é simples. Se $RMg^1 \neq RMg^2$, então, uma unidade de produto poderia ser realocada do mercado com receita marginal mais baixa para o mercado com receita marginal mais alta levando, assim, a um aumento líquido nas receitas e sem aumento no custo (já que o nível de produção permanece constante) e, portanto, os lucros aumentariam. O que implica, então, que na situação inicial o monopolista não estava maximizando os lucros.

Embora simples, a condição (\ref{eq13}) tem uma implicação importante. Note que, no caso de um monopólio, a função receita total é dada por:
\[
R = p(q)q,
\]
portanto, a receita marginal será:
\[
RMg = \frac{dR}{dq} = q\frac{dp}{dq} + p = p\left(1 + \frac{dp}{dq}\frac{q}{p}\right),
\]
seja a elasticidade-preço da demanda dada por:
\[
\epsilon = -\frac{p}{q}\frac{dq}{dp},
\]
temos, então:
\[
RMg = p\left(1 - \frac{1}{\epsilon}\right).
\]

Portanto, pela condição (\ref{eq13}), temos o seguinte resultado:
\begin{eqnarray}
p_1\left(1-\frac{1}{\epsilon_1}\right) &=& p_2\left(1-\frac{1}{\epsilon_2}\right), \nonumber \\
\frac{p_1}{p_2} &=& \frac{1-(1/\epsilon_2)}{1-(1/\epsilon_1)}. \label{eq14}
\end{eqnarray}
Ou seja, \textcolor{blue}{o mercado com maior preço de equilíbrio é aquele com menor elasticidade-preço da demanda no ponto ótimo}.

\subsubsection{Firma monopolista no mercado doméstico e tomadora de preços no mercado mundial}

Suponha uma firma monopolista no mercado doméstico mas que, no mercado mundial, toma o preço unitário ($p_w$) pelo seu produto como dado.

As quantidades vendidas no mercado doméstico e no mercado mundial são, respectivamente, $x_d$ e $x_w$.

O preço obtido no mercado doméstico, como função das vendas realizadas, é dado pela função de demanda inversa:
\[
p_d = P(x_d).
\]

O custo de produção de $x$ unidades é dado pela função custo $C(x)$, independente de como o produto é distribuído entre os mercados doméstico e mundial.

Temos, então, que a receita doméstica total é dada por $R_d = P(x_d)x_d$ e a receita no mercado mundial é igual a $R_w = p_wx_w$. Portanto, a função lucro é igual a\footnote{$P$ e $C$ são funções.}:
\[
\pi(x_d, x_w) = P(x_d)x_d + p_wx_w - C(x_d + x_w).
\]

Supondo que as funções são diferenciáveis, que a firma é maximizadora de lucros e um ponto de máximo interior, então, as condições necessárias de primeira ordem serão:
\begin{eqnarray*}
\pi_d &\therefore& p_d + P'(x_d)x_d - C'(x_d + x_w) = 0, \\
\pi_w &\therefore& p_w - C'(x_d + x_w) = 0.
\end{eqnarray*}

Temos, então, que no mercado mundial o custo marginal deve ser igual ao preço que, por sua vez, também se iguala à receita marginal.

No mercado doméstico, o custo marginal também se iguala à receita marginal. Note, no entanto, que pela função de demanda inversa $P'(x_d) < 0$, portanto, o preço praticado no mercado doméstico ($p_d$) é maior que o custo marginal, ou seja, $p_d > RMg_d = CMg_d$.

Se supormos que a elasticidade-preço da demanda no mercado doméstico é constante e igual a $\epsilon = 2$ segue, então que:
\[
RMg_d = p_d\left(1 - \frac{1}{2}\right) = C'(x_d + x_w) = p_w.
\]
Logo, o preço no mercado doméstico é duas vezes maior que o preço praticado no mercado mundial.
\newpage
\subsection{Duopólio de Cournot}
Considere duas firmas com poder de mercado que produzem produtos idênticos e os vendem em um único mercado descrito pela seguinte função de demanda linear:
\[
p = 100 - (q_1 + q_2).
\]

Assumimos que o custo de produção de cada firma é igual a zero e que o objetivo de cada firma é maximizar seus lucros.

A função lucro da firma $i \in \{1, 2\}$ é, então, dada por:
\[
\pi_i = pq_i = 100q_i - (q_1 + q_2)q_i.
\]

A ideia básica é que o preço de mercado depende da produção total das duas firmas e, então, o lucro de cada firma dependerá do nível de produção da firma rival e da sua própria produção.

Essa forma de interdependência é característica de mercados oligopolistas, dos quais o duopólio de duas firmas é um caso especial.

A dificuldade que uma firma encontra ao resolver seu problema de maximização de lucros é saber o quanto a outra firma irá produzir, já que sem esta informação não conseguirá computar o lucro resultante de qualquer escolha de seu próprio nível de produção.

A hipótese feita pelo economista francês Augustin Cournot é de que cada firma toma o nível de produção da outra firma como um parâmetro no seu processo de escolha de produção e, então, o equilíbrio de mercado será dado pelo par de equações simultâneas resultante.

Neste caso, as condições necessárias de primeira ordem serão:
\begin{eqnarray}
\frac{\partial \pi_1}{\partial q_1}&\therefore& 100 - 2q_1 - q_2 = 0, \nonumber \\
\frac{\partial \pi_2}{\partial q_2}&\therefore& 100 - q_1 - 2q_2 = 0. \nonumber
\end{eqnarray}

Resolvendo esse sistema, temos que $(q_1^*, q_2^*) = (33.33, 33.33)$, a quantidade total produzida será $q = 66,67$ e o preço $p = \$33,33$.

Verificando as condições de segunda ordem, temos:
\[
\frac{\partial^2 \pi_1}{\partial q_1^2} = \frac{\partial^2 \pi_2}{\partial q_2^2} = -2 < 0
\]
logo, temos ponto de máximo relativo.

Portanto, dado a simetria do equilíbrio no exemplo, as firmas irão dividir igualmente o mercado.

Pelas condições de primeira ordem, podemos reescrever as expressões da seguinte maneira:
\[
q_1 = \frac{100-q_2}{2}, \qquad q_2 = \frac{100-q_1}{2}.
\]
Essas equações nos dizem o nível de produção de uma firma que é a melhor estratégia adotada para cada nível de produção possível da outra firma - \textcolor{blue}{melhor resposta} ou \textcolor{blue}{função de reação}.

As funções de melhor resposta de cada firma estão representadas na Figura \ref{fig1} abaixo e o equilíbrio de Cournot é dado pela intersecção entre essas curvas.

\begin{figure}[h!]
    \centering
    \includegraphics[width=0.7\textwidth]{./figures/aula6_fig1.PNG}
    \caption{Equilíbrio de Cournot e funções de reação. Fonte: Hoy et al. (2001).}
    \label{fig1}
\end{figure}

\textbf{O nível total de produção neste mercado duopolista está entre os níveis de monopólio e de competição perfeita}, demonstraremos no próximo tópico.

\subsubsection{Equilíbrio de Cournot com $n$ firmas idênticas}

Considere, agora, um oligopólio com $n$ firmas idênticas que se deparam com a seguinte função de demanda de mercado:
\begin{equation}
    p = 100 - \sum_{i=1}^n q_i.
    \label{eq15}
\end{equation}

Assumindo, ainda, que o custo de produção de cada firma é igual a zero, temos a seguinte função lucro para a firma $i \in \{1, \dots, n\}$:
\begin{equation}
    \pi_i = 100q_i - q_i \sum_{\substack{j = 1 \\j \neq i}}^n q_j - q_i^2.
    \label{eq16}
\end{equation}

A condição necessária de primeira ordem da firma $i$ maximizadora de lucros é dada por:
\begin{equation}
    \frac{\partial \pi_i}{\partial q_i} = 100 - 2q_i - \sum_{\substack{j=1\\j\neq 1}}^n q_j = 0.
    \label{eq17}
\end{equation}

Portanto, a função de reação de cada firma será igual a:
\begin{equation}
    q_i = \frac{100 - \sum_{\substack{j=1\\j\neq 1}}^n q_j}{2}.
    \label{eq18}
\end{equation}

Supondo simetria do equilíbrio de Cournot (um resultado que deveria ser demonstrado) temos que, se $q^*$ é o nível ótimo de produção comum a todas as firmas, então segue da função de reação que:
\[
q^* = \frac{100-(n-1)q^*}{2}.
\]

Portanto, a quantidade ótima produzida em um equilíbrio simétrico de Cournot com $n$ firmas idênticas será:
\begin{equation}
    q^* = \frac{100}{n + 1}.
    \label{eq19}
\end{equation}

\begin{enumerate}
    \item \textcolor{blue}{Monopólio}. No caso de um monopólio temos $n = 1$ e, portanto, a quantidade produzida será igual a:
    \[
    q^* = 50.
    \]
    \item \textcolor{blue}{Duopólio}. Neste caso, $n = 2$ e, portanto:
    \[
    q^* = 33,33\dots
    \]
    
    O que implica que a quantidade total produzida será $2q^* = 66,67$.
    
    \item \textcolor{blue}{Mercado perfeitamente competitivo}. Neste caso temos $n \to \infty$ e, portanto:
    \[
    \lim_{n\to\infty} q^* = \lim_{n\to\infty} \left(\frac{100}{n+1}\right) = 0.
    \]
    
    Ou seja, a quantidade ótima produzida por cada firma individual fica cada vez menor à medida que $n$ aumenta.
    
    A produção total em um mercado perfeitamente competitivo será:
    \[
    \lim_{n\to\infty} nq^* = \lim_{n\to\infty} \left(\frac{100n}{n+1}\right) = \lim_{n\to\infty} \left(\frac{100}{1+\frac{1}{n}}\right) = 100.
    \]
\end{enumerate}

Portanto, como mencionado anteriormente, a quantidade total produzida em um duopólio está entre os níveis totais de produção de um mercado perfeitamente competitivo e um monopólio.
\newpage
\subsection{Diversificação de portfólio}
Uma forma que indivíduos avessos ao risco encontram para reduzir o risco associado às suas decisões de investimento é a \textcolor{blue}{diversificação de portfólio}: ``não colocar todos os ovos em uma única cesta''.

Se distribuirmos os riscos de maneira conveniente entre os ativos, podemos reduzir a variabilidade do portfólio sem reduzir o payoff esperado.

Suponha um indivíduo que dispõe de uma riqueza $W$ para alocar entre dois ativos \textbf{independentes}\footnote{Lembrando que duas variáveis aleatórias $X$ e $Y$ são ditas independentes se a covariância entre elas for nula: $Cov(X,Y) = 0$, ou ainda $E(XY) = E(X)E(Y)$. Note que a inversa não é necessariamente verdadeira, uma covariância igual a zero não implica necessariamente independência estatística.} com risco mas que possuem o mesmo retorno esperado ($\mu$) e variâncias ($\sigma^2$) iguais, ou seja:
\begin{eqnarray}
\mu_1 &=& \mu_2, \label{eq20} \\
\sigma_1^2 &=& \sigma_2^2. \label{eq21}
\end{eqnarray}

\begin{enumerate}
    \item \textcolor{blue}{Portfólio não-diversificado}. Se o indivíduo não diversifica seu portfólio, então, o retorno esperado deste porftólio será igual a $\mu_{ND} = \mu_1 = \mu_2$ e a variância será $\sigma_{ND}^2 = \sigma_1^2 = \sigma_2^2$.
    
    \item \textcolor{blue}{Diversificação de portfólio}. Suponha, agora, que o indivíduo diversifique seu portfólio investindo uma fração $\alpha$ de sua riqueza no ativo 1 e o restante $(1-\alpha)$ no ativo 2.
    
    Neste caso, o retorno esperado do portfólio diversificado será dado por:
    \[
    \mu_D = \alpha \mu_1 + (1-\alpha)\mu_2 = \mu_1 = \mu_2.
    \]
    Ou seja, neste caso, mesmo com a diversificação de portfólio o indivíduo consegue manter o mesmo nível de retorno esperado em seus investimentos.
    
    A variância do portfólio diversificado será dada por:
    \begin{eqnarray*}
    \sigma_D^2 &=& \alpha^2 \sigma_1^2 + (1-\alpha)^2\sigma_2^2 \\
    &=& (1 - 2\alpha + 2\alpha^2)\sigma_1^2.
    \end{eqnarray*}
    
    Portanto, se o objetivo deste indivíduo ao diversificar o portfólio é minimizar o risco associado ao seu investimento, temos que:
    \[
    \min_{\alpha} (1 - 2\alpha + 2\alpha^2)\sigma_1^2
    \]
    
    A condição necessária de primeira ordem será dada por:
    \[
    \frac{d\sigma_D^2}{d\alpha} \therefore -2\sigma_1^2 + 4\alpha\sigma_1^2 = 0.
    \]
    
    Portanto, o ponto crítico associado a este problema de minimização de riscos é dado por $(\alpha, 1-\alpha) = (1/2, 1/2)$.
    
    Note que $\frac{d^2\sigma_D^2}{d\alpha^2} = 4\sigma_1^2 > 0$ e, portanto, o ponto crítico é, de fato, um ponto de mínimo relativo.
    
    Com $\alpha = 1/2$ temos a seguinte relação:
    \[
    \sigma_D^2 = \frac{1}{2}\sigma_1^2 < \sigma_1^2 = \sigma_{ND}^2.
    \]
    \end{enumerate}
    
    O portfólio ótimo divide a riqueza igualmente entre os dois ativos de risco, mantendo o mesmo retorno que o de um portfólio não-diversificado, mas reduzindo a variância pela metade.
    
    Cabe ressaltar que a diversificação funcionou neste exemplo pois assumimos que os ativos de risco são independentes - quando o retorno de um ativo é baixo, existe a chance de que o outro terá um retorno alto, e vice-versa. Portanto, retornos extremos serão compensados algumas vezes, reduzindo a variância associada ao portfólio.
    
    A diversificação funcionará desde que não haja correlação perfeita entre os retornos dos ativos.
    
    Quanto menor a correlação entre os ativos, melhor a diversificação funcionará para reduzir a variância no portfólio.
    
    \newpage
    \subsection{Econometria: Regressão linear}
    A economia empírica preocupa-se com analisar os dados de forma a identificar padrões que ajudem a compreender o passado e, possivelmente, fazer predições sobre o futuro.
    
    Por exemplo, os dados de preço e quantidade para uma mercadoria particular (e.g., gás natural) podem ser usados para tentar estimar uma função demanda.
    
    Essa função de demanda estimada, por sua vez, pode ser usada para fazer predições acerca de como a demanda responderá a variações futuras nos preços.
    
    A técnica mais comumente utilizada para estimar estas funções é a \textcolor{blue}{regressão linear}.
    
    Suponha que uma variável $y$ dependa de uma outra variável $x$ e, ainda, que tenhamos observações $(x_t, y_t)$ de ambas variáveis para $t = 1, 2, \dots, T$.
    
    O modelo simples de regressão linear objetiva ajustar uma função linear do tipo:
    \[
    y = \alpha + \beta x.
    \]
    
    Um ajuste exato só seria possível se existissem valores de $\alpha$ e $\beta$ tais que:
    \[
    y_t = \alpha + \beta x_t, \qquad \forall t = 1, \dots, T,
    \]
    o que é raramente possível.
    
\begin{figure}[h!]
        \centering
        \includegraphics[width=0.9\textwidth]{./figures/aula6_fig2.JPG}
        \caption{Regressão linear: valores ajustados e resíduos. Fonte: SYDSÆTER et al. (2016).}
        \label{fig2}
\end{figure}
    
    Normalmente, então, temos uma equação que relaciona a \textcolor{blue}{variável dependente} $y$ à \textcolor{blue}{variável independente} $x$ da seguinte forma:
    \begin{equation}
        y_t = \alpha + \beta x_t + u_t,
        \label{eq22}
    \end{equation}
    onde $u_t$ é um termo de erro e representa quaisquer outros fatores distintos de $x$ que possam influenciar a variável $y$ - ver Figura \ref{fig2}.
    
    A variável $y$ pode, ainda, ser denominada: variável explicada, variável de resposta, regressando. A variável $x$, por sua vez, pode ser denominada: variável explicativa, variável de controle, regressor.
    
    Nosso objetivo é que o termo de erro seja o menor possível. Uma possível solução seria fazer com que:
    \[
    \sum_{t=1}^T (y_t - \alpha - \beta x_t) = 0.
    \]
    No entanto, neste caso, valores altos e positivos para o termo de erro seriam cancelados por valores altos e negativos. Mesmo que a soma dos erros fosse nula, poderíamos ter uma situação em que nossa aproximação linear estivesse muito longe de fornecer um ajuste razoável aos dados.
    
    Devemos, então, de alguma forma, impedir que grandes desvios positivos cancelem grandes desvios negativos.
    
    Normalmente isso é feito minimizando uma função perda quadrática da forma:
    \begin{equation}
        L(\alpha, \beta) = \sum_{t=1}^T u_t^2 = \sum_{t=1}^T (y_t - \alpha - \beta x_t)^2.
        \label{eq23}
    \end{equation}
    
    Formalmente, então, nosso objetivo é caracterizar as soluções de $\hat{\alpha}$ e $\hat{\beta}$ que resolvem o seguinte problema de minimização irrestrito:
    \begin{equation}
        \min_{\alpha, \beta} \sum_{t=1}^T (y_t - \alpha - \beta x_t)^2,
        \label{eq24}
    \end{equation}
    ou seja, queremos minimizar a soma do quadrado dos resíduos - por isso o método é conhecido como \textcolor{blue}{mínimos quadrados ordinários} (MQO).
    
    As condições necessárias de primeira ordem para obtermos os estimadores de MQO do nosso modelo de regressão linear serão:
    \begin{eqnarray}
    \frac{\partial L(\alpha, \beta)}{\partial \alpha} &\therefore& -2\sum_{t=1}^T (y_t - \hat{\alpha} - \hat{\beta}x_t) = 0, \label{eq25} \\
    \frac{\partial L(\alpha, \beta)}{\partial \beta} &\therefore& -2\sum_{t=1}^T x_t(y_t - \hat{\alpha} - \hat{\beta}x_t) = 0. \label{eq26}
    \end{eqnarray}
    
    Da equação (\ref{eq25}) temos que:
    \begin{equation}
    \hat{\alpha} = \mu_y - \hat{\beta}\mu_x,
    \label{eq27}
    \end{equation}
    onde $\mu_x = \frac{\sum_{t=1}^T x_t}{T}$ e $\mu_y = \frac{\sum_{t=1}^T y_t}{T}$.
    
    Substituindo a expressão anterior em (\ref{eq26}) temos:
    \begin{eqnarray*}
    \sum_{t=1}^T x_t[(y_t - \mu_y) - \hat{\beta}(x_t - \mu_x)] &=& 0 \\
    \sum_{t=1}^T x_t(y_t - \mu_y) &=& \hat{\beta}\sum_{t=1}^T x_t(x_t - \mu_x) \\
    \sum_{t=1}^T (x_t - \mu_x)(y_t - \mu_y) &=& \hat{\beta}\sum_{t=1}^T (x_t - \mu_x)^2
    \end{eqnarray*}
    
    Portanto, temos que:
    \begin{equation}
        \hat{\beta} = \frac{\sum_{t=1}^T (x_t - \mu_x)(y_t - \mu_y)}{\sum_{t=1}^T (x_t - \mu_x)^2} = \frac{\widehat{Cov(x, y)}}{\hat{\sigma}_x^2} = \hat{\rho}_{xy}\frac{\hat{\sigma_y}}{\hat{\sigma_x}}.
        \label{eq28}
    \end{equation}
    
    Uma vez calculado o valor de $\hat{\beta}$ como a razão entre a covariância amostral de $x$ e $y$ e a variância amostral de $x$, podemos obter $\hat{\alpha}$ pela expressão (\ref{eq27}).
\newpage    
\subsection{Senhoriagem e inflação}
Inflação, às vezes, atinge níveis consideravelmente elevados. Os casos mais extremos são denominados \textcolor{blue}{hiperinflações} que, tecnicamente, são definidos como períodos quando a taxa de inflação supera 50\% ao mês. O recorde histórico hiperinflacionário foi registrado na Hungria entre agosto de 1945 e julho de 1946. Durante esse período, o nível de preços aumentou por um fator de aproximadamente $10^{27}$. No mês de pico hiperinflacionário, os preços, em média, triplicavam diariamente.

A hiperinflação no Zimbábue entre 2007 e 2009 foi quase tão elevada, com os preços duplicando diariamente.

A existência de um trade-off entre produto e inflação não é suficiente para explicar períodos hiperinflacionários ou mesmo níveis muito elevados de inflação. Quando a inflação atinge três dígitos, os custos inflacionários são consideravelmente elevados e os efeitos reais de variações monetários, muito provavelmente, serão muito limitados. Portanto, nenhum formulador de política econômica optaria por submeter uma economia a custos tão elevados para obter ganhos tão modestos de produto.

A causa subjacente da maioria, se não todos, dos episódios de hiperinflação  ou inflação muito elevada é a necessidade do governo de obter \textcolor{blue}{senhoriagem} - isto é, as receitas geradas por imprimir moeda.

Guerras, quedas nos preços de bens de exportação, evasão fiscal, etc. frequentemente levam governos a situações de elevados déficits orçamentários. E muitas das vezes os investidores não tem a confiança de que o governo irá cumprir suas obrigações com relação à dívida pública e, portanto, não estarão dispostos a comprar títulos - financiar o governo. Portanto, a única escolha do governo é recorrer às receitas de senhoriagem.

Aqui analisaremos a interação entre necessidade de senhoriagem, crescimento da moeda e inflação. Focaremos apenas na situação em que as necessidades de senhoriagem são sustentáveis e veremos como isso pode levar a elevados níveis de inflação. Não estudaremos o caso em que as necessidades de senhoriagem são insustentáveis e, portanto, podendo levar a hiperinflações.

\subsubsection{Taxa de inflação e senhoriagem}

Assumiremos que a função de demanda por encaixes reais depende negativamente da taxa nominal de juros e positivamente da renda real:
\begin{eqnarray}
\frac{M}{P} &=& L(i, Y), \label{eq29} \\
&=& L(r + \pi^e, Y), \qquad L_i < 0, \quad L_Y>0. \nonumber
\end{eqnarray}
Como estamos interessados na receita obtida pelo governo ao criar moeda, $M$ deve ser interpretado como base monetária\footnote{\emph{High-powered money}.} (moeda e reservas emitidas pelo governo).

Vamos focar no estado estacionário. Portanto, é razoável assumir que o produto real e a taxa de juros real não são afetados pela taxa de crescimento da moeda, e que a inflação efetiva é igual à inflação esperada.

Se desconsiderarmos o crescimento do produto, então, no estado estacionário, a quantidade de encaixes reais é constante. Isso implica que a taxa de inflação é igual à taxa de crescimento da moeda:
\begin{equation}
    \frac{M}{P} = L(\bar{r} + g_M, \bar{Y}),
    \label{eq30}
\end{equation}
$g_M$ é a taxa de crescimento da moeda $\dot{M}/M$.

A quantidade de aquisições reais por unidade do tempo que o governo financia via criação de moeda é igual ao aumento do estoque nominal de moeda por unidade do tempo dividido pelo nível de preços:
\begin{eqnarray}
S &=& \frac{\dot{M}}{P} \nonumber \\
&=& \frac{\dot{M}}{M}\frac{M}{P} \nonumber \\
&=& g_M \frac{M}{P}. \label{eq31}
\end{eqnarray}
Ou seja, no estado estacionário, a receita real de senhoriagem é igual à taxa de crescimento do estoque de moeda multiplicada pela quantidade de encaixes reais. A taxa de crescimento da moeda, como vimos, é igual à taxa à qual a retenção nominal de moeda perde valor real, $\pi$. Portanto, a senhoriagem é igual à ``alíquota de imposto'' sobre os encaixes reais, $\pi$, vezes a quantidade sujeita à taxação, $M/P$. Por esse motivo, as receitas de senhoriagem são frequentemente chamadas de receitas de \textcolor{blue}{imposto inflacionário}.

Substituindo (\ref{eq31}) em (\ref{eq30}), temos:
\begin{equation}
    S = g_M L(\bar{r} + g_M, \bar{Y}).
    \label{eq32}
\end{equation}
A equação (\ref{eq32}) nos diz que um aumento no crescimento da moeda $g_M$ aumenta a senhoriagem ao aumentar a taxa na qual as detenções reais de moeda são taxadas, mas diminui a senhoriagem ao reduzir a base tributária. Formalmente:
\begin{equation}
    \frac{dS}{dg_M} = L(\bar{r} + g_M, \bar{Y}) + g_ML_i(\bar{r} + g_M, \bar{Y}).
\end{equation}

O primeiro termo é positivo enquanto o segundo é negativo. O segundo termo tende a zero quando $g_M \to 0$. Como o primeiro termo é estritamente positivo, segue que para valores suficientemente baixos de $g_M$, $dS/dg_M$ é positivo: a baixas alíquotas de imposto, a senhoriagem cresce quando a alíquota aumenta. É plausível, no entanto, que à medida que $g_M$ aumenta, o segundo termo eventualmente venha a dominar - quando a alíquota é muito alta, se continuar crescendo a receita diminuirá. Temos, portanto, um caso de curva de Laffer e imposto inflacionário como na Figura \ref{fig3}.

\begin{figure}[h!]
    \centering
    \includegraphics[width=0.7\textwidth]{./figures/aula6_fig3.PNG}
    \caption{A curva de Laffer de imposto inflacionário. Fonte: Romer (2018).}
    \label{fig3}
\end{figure}

Em 1956, Cagan sugeriu que uma boa descrição para a função de demanda por moeda, particularmente em períodos de alta inflação, é dada por:
\begin{equation}
    \ln \frac{M}{P} = a - bi + \ln Y, \qquad b > 0.
    \label{eq34}
\end{equation}

Transformando a expressão anterior em nível e substituindo na equação (\ref{eq32}), temos:
\begin{eqnarray}
S &=& g_M e^a\bar{Y}e^{-b(\bar{r} + g_M)} \nonumber \\
&=& C g_M e^{-bg_M}, \label{eq35}
\end{eqnarray}
onde $C \equiv e^a\bar{Y}e^{-b\bar{r}}$.

O impacto de uma variação no crescimento da moeda sobre a receita de senhoriagem é, portanto:
\begin{eqnarray}
    \frac{dS}{dg_M} &=& Ce^{-bg_M} - bCg_Me^{-bg_M} \nonumber \\
    &=& (1-bg_M)Ce^{-bg_M}. \label{eq36}
\end{eqnarray}
A expressão é positiva para $g_M<1/b$ e negativa caso contrário - Figura \ref{fig4}.

\begin{figure}[h!]
    \centering
    \includegraphics[width=0.7\textwidth]{./figures/aula6_fig4.PNG}
    \caption{Necessidade de senhoriagem e determinação da inflação. Fonte: Romer (2018).}
    \label{fig4}
\end{figure}

A receita com senhoriagem atinge seu valor máximo (pela CPO) quando $g_M = 1/b$. Cagan, Sachs e Larrain (1993) argumentam que para a maioria dos países, a senhoriagem no pico da curva de Laffer é de cerca de 10\% do PIB.

Consideremos, então, um governo que precise financiar uma quantidade $G$ de aquisições reais via senhoriagem. Vamos assumir que $G < S^{MAX}$ - a quantidade máxima factível de receita de senhoriagem. Portanto, como evidenciado pela Figura \ref{fig4}, existem duas taxas de crescimento de moeda que são compatíveis com essa necessidade de financiamento - $g_1$ e $g_2$.

Na primeira, a inflação é baixa e os encaixes reais são elevados. Na segunda, a inflação é elevada e os encaixes reais são baixos. O equilíbrio de inflação elevada tem algumas características peculiares de estática comparativa, e.g., uma redução na necessidade de senhoriagem do governo aumenta a inflação. Como não parece ser esse o caso observado no mundo real, focaremos no equilíbrio de baixa-inflação e, portanto, a taxa de crescimento da moeda (e taxa de inflação) será dada por $g_1$.

Essa análise fornece uma explicação para taxas de inflação elevadas: a necessidade do governo por senhoriagem. Suponha, por exemplo, $b = 1/3$ e $S^{MAX} = 10\%$ do PIB - compatíveis com estimativas empíricas. Vimos que a senhoriagem é máxima quando $g_M = 1/b$ e, portanto, $S^{MAX} = Ce^{-1}/b$. Então, $C$ deve ser aproximadamente 9\% do PIB.

Neste caso, então, um aumento de 2\% do PIB de senhoriagem requer $g_M \approx 0,24$, um aumento de 5\% requer $g_M \approx 0,70$ e aumentar 8\% requer $g_m \approx 1,42$. Portanto, necessidades moderadas de senhoriagem podem levar a inflações substanciais, e necessidades de senhoriagem elevadas produzem inflações altas.

\subsection{Bibliografia}
\begin{itemize}
    \item CHIANG, A.C.; WAINWRIGHT, K. Matemática para economistas. Rio de Janeiro: Elsevier, 2006.
    \item HOY, M.; LIVERNOIS, J.; McKENNA, C.; REES, R.; STENGOS, T. Mathematics for Economics. 2.ed., Massachussetts: MIT Press, 2001.
    \item ROMER, D. Advanced Macroeconomics. 5th.ed. New York: McGraw-Hill Companies, 2018.
    \item SIMON, C.P.; BLUME, L. Matemática para economistas. Porto Alegre: Bookman, 2004.
    \item SYDSÆTER, K.; HAMMOND, P.J.; STRØM, A.; CARVAJAL, A. Essential mathematics for economic analysis. 5th.ed. Harlow, UK: Pearson Education Limited, 2016.
\end{itemize}
\end{document}