\documentclass[preprintnumbers,nofootinbib,amsmath,amssymb,12pt]{article}
\usepackage{dcolumn}% Align table columns on decimal point
\usepackage{bm}% bold math
\usepackage{amsthm}
\usepackage[english,brazilian]{babel}
\usepackage[utf8]{inputenc}
\usepackage{makeidx}
\usepackage{framed}
\usepackage{cmap}
\usepackage{graphicx}
\usepackage{float}
\usepackage[caption = false]{subfig}
\usepackage[T1]{fontenc}
\usepackage{amsmath,amssymb,amsfonts,textcomp}
\usepackage{setspace}
\usepackage{caption}
\usepackage{multicol}
\usepackage{multirow}
\usepackage{array}
%\usepackage{booktabs}
\usepackage{setspace}
\usepackage{lscape}
\usepackage{anysize}
\usepackage{hyperref}
\usepackage{booktabs,caption}
\usepackage{longtable}
\usepackage{chngcntr}
\usepackage[portuguese,linesnumbered,ruled]{algorithm2e}
\hypersetup{colorlinks=true, linkcolor=blue, citecolor=blue, filecolor=blue, pagecolor=blue, urlcolor=blue,
            pdfauthor={Nome do Autor},
            pdftitle={Título do Projeto},
            pdfsubject={Assunto do Projeto},
            pdfkeywords={palavra-chave, palavra-chave, palavra-chave},
            pdfproducer={Latex},
            pdfcreator={pdflatex}}
\marginsize{2cm}{2cm}{2cm}{2cm}
\renewenvironment{quote}{\list{}{\small \singlespacing \leftmargin=4cm\rightmargin=0cm}\item[]}{\endlist} 
\usepackage{natbib}
\newtheorem{alg}{Algoritmo}
\newtheorem{prop}{Proposição}
\newtheorem{teo}{Teorema}
\newtheorem{defi}{Definição}
\newtheorem{lema}{Lema}
\newtheorem{coro}{Corolário}
\newtheorem{hip}{Hipótese}
\newcommand{\algorithmfootnote}[2][\footnotesize]{%
  \let\old@algocf@finish\@algocf@finish% Store algorithm finish macro
  \def\@algocf@finish{\old@algocf@finish% Update finish macro to insert "footnote"
    \leavevmode\rlap{\begin{minipage}{\linewidth}
    #1#2
    \end{minipage}}%
  }%
}

\title{}
\author{}
\date{}

\begin{document}

\begin{center}
\textbf{UNIVERSIDADE DO ESTADO DE SANTA CATARINA \\ Centro de Ciências da Administração e Socioeconômicas \\ Departamento de Ciências Econômicas}
\end{center}

\vskip 1em

\noindent\textbf{Disciplina:} Métodos Quantitativos em Economia I

\noindent\textbf{Docente:} \href{https://pvfonseca.github.io}{Paulo Victor da Fonseca} 

\noindent\textbf{Contato:} \href{mailto:paulo.fonseca@udesc.br}{paulo.fonseca@udesc.br}

\noindent\textbf{Página da disciplina:} \href{https://pvfonseca.github.io/teaching/metodosquant}{Métodos Quantitativos I}

\noindent\textbf{Data de entrega:} 28/09/2023

\vskip 1em

\noindent\textbf{Discente:} \hrulefill

\vskip 1em
\begin{enumerate}
    \item Encontre e classifique os pontos críticos (máximo local, mínimo local, ou nenhum desses casos) de cada uma das funções a seguir.
    \begin{enumerate}
        \item $f(x,y) = x^2 + xy + 2y^2 + 3$.
        \item $f(x,y) = -x^2 - y^2 + 6x + 2y$.
        \item $f(x,y) = 2x^3 + xy^2 + 5x^2 + y^2$.
        \item $f(x,y) = e^{2x} - 2x + 2y^2 + 3$.
        \item $f(x,y) = e^{2x}(x+y^2+2y)$.
        \item $f(x,y,z) = xz + x^2 - y + yz + y^2 + 3z^2$.
    \end{enumerate}
    
    \item Uma firma é um produtor em um mercado perfeitamente competitivo e vende dois bens $G_1$ e $G_2$ a \$1000 e \$800, respectivamente. O custo total de produção destes bens é dado por:
    \[
    CT = 2Q_1^2 + 2Q_1Q_2 + Q_2^2,
    \]
    onde $Q_1$ e $Q_2$ denotam o nível de produção de $G_1$ e $G_2$, respectivamente.
    
    Encontre o lucro máximo e os valores de $Q_1$ e $Q_2$ aos quais este lucro é atingido. Mostre que este ponto é, de fato, um ponto de máximo.
    
    \item Uma firma tem a possibilidade de cobrar preços distintos de seu produto no mercado doméstico e externo. As equações de demanda correspondentes são dadas por:
    \begin{eqnarray*}
    Q_1 &=& 300 - P_1 \\
    Q_2 &=& 400 - 2P_2.
    \end{eqnarray*}
    A função custo total é dada por:
    \[
    CT = 5000 + 100Q,
    \]
    onde $Q = Q_1 + Q_2$.
    
    Determine os preços que esta firma deve cobrar para maximizar seus lucros com discriminação de preços e calcule o valor deste lucro. Mostre que este ponto é, de fato, um ponto de máximo.
    
    \item Considere as funções abaixo e classifique se são funções côncavas, convexas, estritamente côncavas, estritamente convexas ou nenhuma delas. Além disso, encontre seus pontos extremos e determine sua natureza.
    \begin{enumerate}
        \item $z = (x + y)^2$.
        \item $z = (x-2)^2 + (y-5)^2 - 3$.
        \item $z = x^2 + xy + y^2 + \frac{2}{x} + \frac{2}{y}$.
        \item $z = xexp(-x^2 - y^2)$.
    \end{enumerate}
    
    \item Considere os conjuntos a seguir e determine se o conjunto é convexo ou não (a resolução pode ser feita pela definição de conjunto convexo ou argumentando pelo gráfico dos conjuntos).
    \begin{enumerate}
        \item $\{(x,y)|y\geq 2x-x^2; x>0, y>0\}$.
        \item $\{(x,y)|y\leq \ln(x)\}$.
        \item $\{(x,y)| y = -e^x$\}.
        \item $\{(x,y)| y \geq -e^x$\}.
    \end{enumerate}
    \end{enumerate}
\end{document}